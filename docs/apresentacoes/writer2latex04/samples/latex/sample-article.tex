% This file was converted to LaTeX by Writer2LaTeX ver. 0.4
% see http://www.hj-gym.dk/~hj/writer2latex for more info
\documentclass[12pt]{article}
\usepackage[ascii]{inputenc}
\usepackage[T1]{fontenc}
\usepackage[english]{babel}
\usepackage{amsmath,amssymb,amsfonts,textcomp}
\usepackage{color}
\usepackage{multicol}
\usepackage{longtable}
\usepackage{makeidx}
\usepackage{hyperref}
\hypersetup{colorlinks=true, linkcolor=blue, filecolor=blue, pagecolor=blue, urlcolor=blue}
\setlength\tabcolsep{1mm}
\renewcommand\arraystretch{1.3}
\makeindex
\newtheorem{theorem}{Theorem}
\begin{document}
\title{Sample article}

\author{Henrik Just}

\date{July 2004}

\maketitle
\renewcommand\abstractname{Abstract}

\begin{abstract}
This sample illustrates how to create a LaTeX article based on the
template \texttt{LaTeX{}-article.stw}. When you do this, your job is to
concentrate on the \emph{content}, not on the \emph{formatting}.

\end{abstract}
\setcounter{tocdepth}{5}
\renewcommand\contentsname{Table of Contents}
\tableofcontents
\part{This article has two parts}
This part is the first! To insert a new part, choose the style
\textbf{part}.

\section{\label{ref:example}Various types of block content}
This is the first section. To insert a new section, choose the style
\textbf{Heading 2}.

\subsection{Lists}
This is the first subsection. To insert a new subsection, choose the
style \textbf{Heading 3}.

\subsubsection{Enumerated lists}
This is the first subsubsection. To insert a new subsubsection, choose
the style \textbf{Heading 4}.

\begin{enumerate}
\item To insert an enumerated list, choose the style \textbf{enumerate}.
\item An enumerated list can have several items.

\begin{enumerate}
\item This includes subitems

\begin{enumerate}
\item in fact

\begin{enumerate}
\item up to four levels
\end{enumerate}
\item is possible
\end{enumerate}
\item going back...
\end{enumerate}
\item back at top level.
\end{enumerate}
\subsubsection{Itemized lists}
This subsection shows how to create an itemized list.

\begin{itemize}
\item To insert an itemized list, choose the style \textbf{itemize}.

\begin{itemize}
\item an itemized list can have subitems as well
\end{itemize}
\item but this list is quite short.
\end{itemize}
\subsubsection{Description lists}
\begin{description}
\item[Item header]

To insert a description list item, choose the style \textbf{List
Heading}. The header for the description item goes there.

\item[Item content]

The next paragraph will automatically be in the style \textbf{List
Contents}. The content for the description item goes there. The content
can span several paragraphs.

\end{description}
\subsection{Tables}
You can insert tables as usual:

\begin{longtable}[l]{|p{3.799cm}|p{3.799cm}|p{3.797cm}|}
\hline
\textbf{This}
&
\textbf{is a}
&
\textbf{table}
\\\hline
\endhead
Borders are supported.
&
Horizontally merged cells works as you would expect...
&
... but try to avoid merging cells vertically, as this will give bad
results.
\\\hline
\multicolumn{1}{p{3.799cm}|}{No borders at this corner!
}&
\multicolumn{2}{p{7.7960005cm}|}{This cell spans two columns. Some more
text to show this is true.
}\\\cline{2-3}
\end{longtable}
\subsection{Multiple columns}
\begin{multicols}{2}
This is a section with multiple columns. You can insert such a section
as usual (\textbf{Insert {--} Section}).

Some more text to illustrate that there are two columns. Some more text
to illustrate that there are two columns. Some more text to illustrate
that there are two columns. Some more text to illustrate that there are
two columns.
\end{multicols}
\subsection{Special environments}
\subsubsection{Flushleft, center and flushright}
\begin{center}
These paragraphs are centered. To insert a centered paragraph, select
the style \textbf{center}. Likewise, you can select the styles
\textbf{flushleft} and \textbf{flushright} for left justified and right
justified paragraphs. Some more text to illustrate, that the paragraph
is centered.

This is another centered paragraph. Some more text to illustrate, that
the paragraph is centered. Some more text to illustrate, that the
paragraph is centered.

\end{center}
\subsubsection{Verse, quote and quotation}
These are other special types of paragraphs. For example the following
paragraph is formatted as a quote:

\begin{quote}
A quote is indented on both sides. Some more text to illustrate the
effect. Some more text to illustrate the effect. Some more text to
illustrate the effect. Some more text to illustrate the effect.

\end{quote}
This is ordinary text body.

\subsubsection{Preformatted text (aka verbatim text)}
\begin{verbatim}
The style Preformatted text has a special purpose:
It is used for text that is formatted using
     spaces        and
linebreaks.
You can compare this to text written on a typewriter
(and the font used will be a typewriter style font).
       Fruit    Color
       ------   -----
       Banana   Yellow  <--- this formatting will work!
       Apple    Red
       Orange   Orange
No automatic linebreak will occur.
Footnotes etc. are not allowed in Preformatted text.
\end{verbatim}
\section{Various types of inline content}
Of course footnotes\footnote{\label{fnt:ftn0}This is a footnote} etc.
are inserted as usual. Some other examples:

A referece to section 1 on page \pageref{ref:example}, another reference
to footnote number \ref{fnt:ftn0} on page \pageref{fnt:ftn0}.

A hyperlink
\href{http://www.openoffice.org/}{http://www.openoffice.org}.

The current page number is \thepage{}.

\part{Another part of the article}
This part contains some final expamples.

\section{Bibliograhy}
This is a bibliograhic reference to \cite{SAU99}, and this one is for
\cite{MEA99A}. Another reference to \cite{SAU99}.

\section{Alphabetical index}
Here are some words for the alphabetical index: \index{banana}banana,
\index{apple}apple, \index{orange}orange.

\bibliographystyle{plain}
\bibliography{sample-article}
\printindex
\end{document}
