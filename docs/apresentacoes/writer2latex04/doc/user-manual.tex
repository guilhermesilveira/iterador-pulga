% This file was converted to LaTeX by Writer2LaTeX ver. 0.4
% see http://www.hj-gym.dk/~hj/writer2latex for more info
\documentclass[12pt]{article}
\usepackage[ascii]{inputenc}
\usepackage[T1]{fontenc}
\usepackage[english]{babel}
\usepackage[geometry,weather,misc,clock]{ifsym}
\usepackage{pifont}
\usepackage{eurosym}
\usepackage{amsmath,wasysym,amssymb,amsfonts,textcomp}
\usepackage{color}
\usepackage{longtable}
\usepackage{hyperref}
\hypersetup{pdftex, colorlinks=true, linkcolor=blue, filecolor=blue, pagecolor=blue, urlcolor=blue, pdfauthor=Henrik Just}
\usepackage[pdftex]{graphicx}
% Text styles
\newcommand\textstyleBulletSymbols[1]{{\fontsize{9pt}{10.8pt}\selectfont \textrm{#1}}}
\newcommand\textstyleTeletype[1]{\texttt{#1}}
\newcommand\textstyleNumberingSymbols[1]{#1}
\newcommand\textstyleSourceText[1]{\texttt{#1}}
\newcommand\textstyleUserEntry[1]{\texttt{#1}}
\newcommand\textstyleFootnoteanchor[1]{\textsuperscript{#1}}
\raggedbottom
% Headings and outline numbering
\makeatletter
\renewcommand\section{\@startsection{section}{1}{0cm}{0.423cm}{0.212cm}{\clearpage\normalfont\normalsize\normalcolor\fontsize{16.1pt}{19.32pt}\selectfont\sffamily\bfseries}}
\renewcommand\subsection{\@startsection{subsection}{2}{0cm}{0.423cm}{0.212cm}{\normalfont\normalsize\normalcolor\fontsize{14pt}{16.8pt}\selectfont\sffamily\bfseries}}
\renewcommand\subsubsection{\@startsection{subsubsection}{3}{0cm}{0.423cm}{0.212cm}{\normalfont\normalsize\normalcolor\fontsize{12pt}{14.4pt}\selectfont\sffamily\bfseries}}
\renewcommand\@seccntformat[1]{\csname @textstyle#1\endcsname{\csname the#1\endcsname}\csname @distance#1\endcsname}
\setcounter{secnumdepth}{2}
\newcommand\@distancesection{}
\newcommand\@textstylesection[1]{#1}
\renewcommand\thesection{\arabic{section} }
\newcommand\@distancesubsection{}
\newcommand\@textstylesubsection[1]{#1}
\renewcommand\thesubsection{\arabic{section}.\arabic{subsection} }
\newcommand\@distancesubsubsection{}
\newcommand\@textstylesubsubsection[1]{#1}
\makeatother
% Paragraph styles
\renewcommand\familydefault{\rmdefault}
\renewcommand\normalcolor{\color{black}}
\newenvironment{stylePi}{\clearpage\setlength\leftskip{0cm plus 1fil}\setlength\rightskip{0cm plus 1fil}\setlength\parindent{0cm}\setlength\parfillskip{0pt}\setlength\parskip{0cm}\writerlistparindent\writerlistleftskip\leavevmode\normalfont\normalsize\normalcolor\fontsize{18pt}{21.6pt}\selectfont\sffamily\bfseries\writerlistlabel\ignorespaces}{\unskip\vspace{0.199cm}\par}
\newenvironment{stylePii}{\setlength\leftskip{0cm plus 1fil}\setlength\rightskip{0cm plus 1fil}\setlength\parindent{0cm}\setlength\parfillskip{0pt}\setlength\parskip{0cm}\writerlistparindent\writerlistleftskip\leavevmode\normalfont\normalsize\normalcolor\fontsize{20pt}{24.0pt}\selectfont\sffamily\bfseries\writerlistlabel\ignorespaces}{\unskip\vspace{0.3cm}\par}
\newenvironment{stylePiii}{\setlength\leftskip{0cm plus 1fil}\setlength\rightskip{0cm plus 1fil}\setlength\parindent{0cm}\setlength\parfillskip{0pt}\setlength\parskip{0cm}\writerlistparindent\writerlistleftskip\leavevmode\normalfont\normalsize\normalcolor\fontsize{18pt}{21.6pt}\selectfont\sffamily\bfseries\writerlistlabel\ignorespaces}{\unskip\vspace{0.3cm}\par}
\newenvironment{stylePiv}{\setlength\leftskip{0cm plus 1fil}\setlength\rightskip{0cm plus 1fil}\setlength\parindent{0cm}\setlength\parfillskip{0pt}\setlength\parskip{0.499cm}\writerlistparindent\writerlistleftskip\leavevmode\normalfont\normalsize\normalcolor\fontsize{14pt}{16.8pt}\selectfont\sffamily\bfseries\writerlistlabel\ignorespaces}{\unskip\vspace{0cm}\par}
\newenvironment{stylePv}{\setlength\leftskip{0cm plus 1fil}\setlength\rightskip{0cm plus 1fil}\setlength\parindent{0cm}\setlength\parfillskip{0pt}\setlength\parskip{0.199cm}\writerlistparindent\writerlistleftskip\leavevmode\normalfont\normalsize\normalcolor\fontsize{14pt}{16.8pt}\selectfont\sffamily\bfseries\writerlistlabel\ignorespaces}{\unskip\vspace{0cm}\par\clearpage}
\newenvironment{styleTextbody}{\setlength\leftskip{0cm}\setlength\rightskip{0cm}\setlength\parindent{0cm}\setlength\parfillskip{0pt plus 1fil}\setlength\parskip{0.101cm}\writerlistparindent\writerlistleftskip\leavevmode\normalfont\normalsize\normalcolor\writerlistlabel\ignorespaces}{\unskip\vspace{0.21cm}\par}
\newenvironment{stylePvii}{\setlength\leftskip{0cm}\setlength\rightskip{0cm}\setlength\parindent{0cm}\setlength\parfillskip{0pt plus 1fil}\setlength\parskip{0.101cm}\writerlistparindent\writerlistleftskip\leavevmode\normalfont\normalsize\normalcolor\writerlistlabel\ignorespaces}{\unskip\vspace{0.21cm}\par}
\newenvironment{stylePviii}{\setlength\leftskip{0cm}\setlength\rightskip{0cm}\setlength\parindent{0cm}\setlength\parfillskip{0pt plus 1fil}\setlength\parskip{0.101cm}\writerlistparindent\writerlistleftskip\leavevmode\normalfont\normalsize\normalcolor\writerlistlabel\ignorespaces}{\unskip\vspace{0.21cm}\par}
\newenvironment{stylePix}{\setlength\leftskip{0cm}\setlength\rightskip{0cm}\setlength\parindent{0cm}\setlength\parfillskip{0pt plus 1fil}\setlength\parskip{0.101cm}\writerlistparindent\writerlistleftskip\leavevmode\normalfont\normalsize\normalcolor\writerlistlabel\ignorespaces}{\unskip\vspace{0.21cm}\par}
\newenvironment{stylePx}{\setlength\leftskip{1cm}\setlength\rightskip{0cm}\setlength\parindent{0cm}\setlength\parfillskip{0pt plus 1fil}\setlength\parskip{0cm}\writerlistparindent\writerlistleftskip\leavevmode\normalfont\normalsize\normalcolor\ttfamily\writerlistlabel\ignorespaces}{\unskip\vspace{0.101cm}\par}
\newenvironment{stylePxi}{\setlength\leftskip{0cm}\setlength\rightskip{0cm}\setlength\parindent{0cm}\setlength\parfillskip{0pt plus 1fil}\setlength\parskip{0.101cm}\writerlistparindent\writerlistleftskip\leavevmode\normalfont\normalsize\normalcolor\writerlistlabel\ignorespaces}{\unskip\vspace{0.21cm}\par}
\newenvironment{stylePxii}{\setlength\leftskip{0cm}\setlength\rightskip{0cm}\setlength\parindent{0cm}\setlength\parfillskip{0pt plus 1fil}\setlength\parskip{0.101cm}\writerlistparindent\writerlistleftskip\leavevmode\normalfont\normalsize\normalcolor\writerlistlabel\ignorespaces}{\unskip\vspace{0.21cm}\par}
\newenvironment{stylePxiii}{\setlength\leftskip{0cm}\setlength\rightskip{0cm}\setlength\parindent{0cm}\setlength\parfillskip{0pt plus 1fil}\setlength\parskip{0.101cm}\writerlistparindent\writerlistleftskip\leavevmode\normalfont\normalsize\normalcolor\writerlistlabel\ignorespaces}{\unskip\vspace{0.21cm}\par}
\newenvironment{stylePxiv}{\setlength\leftskip{0cm}\setlength\rightskip{0cm}\setlength\parindent{0cm}\setlength\parfillskip{0pt plus 1fil}\setlength\parskip{0.101cm}\writerlistparindent\writerlistleftskip\leavevmode\normalfont\normalsize\normalcolor\writerlistlabel\ignorespaces}{\unskip\vspace{0.21cm}\par}
\newenvironment{stylePxv}{\setlength\leftskip{1cm}\setlength\rightskip{0cm}\setlength\parindent{0cm}\setlength\parfillskip{0pt plus 1fil}\setlength\parskip{0cm}\writerlistparindent\writerlistleftskip\leavevmode\normalfont\normalsize\normalcolor\ttfamily\writerlistlabel\ignorespaces}{\unskip\vspace{0.101cm}\par}
\newenvironment{stylePxvi}{\setlength\leftskip{0cm}\setlength\rightskip{0cm}\setlength\parindent{0cm}\setlength\parfillskip{0pt plus 1fil}\setlength\parskip{0.101cm}\writerlistparindent\writerlistleftskip\leavevmode\normalfont\normalsize\normalcolor\writerlistlabel\ignorespaces}{\unskip\vspace{0.21cm}\par}
\newenvironment{stylePxvii}{\setlength\leftskip{1cm}\setlength\rightskip{0cm}\setlength\parindent{0cm}\setlength\parfillskip{0pt plus 1fil}\setlength\parskip{0cm}\writerlistparindent\writerlistleftskip\leavevmode\normalfont\normalsize\normalcolor\ttfamily\writerlistlabel\ignorespaces}{\unskip\vspace{0.101cm}\par}
\newenvironment{stylePxviii}{\setlength\leftskip{0cm}\setlength\rightskip{0cm}\setlength\parindent{0cm}\setlength\parfillskip{0pt plus 1fil}\setlength\parskip{0.101cm}\writerlistparindent\writerlistleftskip\leavevmode\normalfont\normalsize\normalcolor\writerlistlabel\ignorespaces}{\unskip\vspace{0.21cm}\par}
\newenvironment{stylePreformattedText}{\setlength\leftskip{1cm}\setlength\rightskip{0cm}\setlength\parindent{0cm}\setlength\parfillskip{0pt plus 1fil}\setlength\parskip{0cm}\writerlistparindent\writerlistleftskip\leavevmode\normalfont\normalsize\normalcolor\ttfamily\writerlistlabel\ignorespaces}{\unskip\vspace{0.101cm}\par}
\newenvironment{stylePxix}{\setlength\leftskip{0cm}\setlength\rightskip{0cm}\setlength\parindent{0cm}\setlength\parfillskip{0pt plus 1fil}\setlength\parskip{0.101cm}\writerlistparindent\writerlistleftskip\leavevmode\normalfont\normalsize\normalcolor\writerlistlabel\ignorespaces}{\unskip\vspace{0.21cm}\par}
\newenvironment{stylePxx}{\setlength\leftskip{0cm}\setlength\rightskip{0cm}\setlength\parindent{0cm}\setlength\parfillskip{0pt plus 1fil}\setlength\parskip{0.101cm}\writerlistparindent\writerlistleftskip\leavevmode\normalfont\normalsize\normalcolor\writerlistlabel\ignorespaces}{\unskip\vspace{0.21cm}\par}
\newenvironment{stylePxxi}{\setlength\leftskip{0cm}\setlength\rightskip{0cm}\setlength\parindent{0cm}\setlength\parfillskip{0pt plus 1fil}\setlength\parskip{0.101cm}\writerlistparindent\writerlistleftskip\leavevmode\normalfont\normalsize\normalcolor\writerlistlabel\ignorespaces}{\unskip\vspace{0.21cm}\par}
\newenvironment{stylePxxii}{\setlength\leftskip{0cm}\setlength\rightskip{0cm}\setlength\parindent{0cm}\setlength\parfillskip{0pt plus 1fil}\setlength\parskip{0.101cm}\writerlistparindent\writerlistleftskip\leavevmode\normalfont\normalsize\normalcolor\writerlistlabel\ignorespaces}{\unskip\vspace{0.21cm}\par}
\newenvironment{stylePxxiii}{\setlength\leftskip{0cm}\setlength\rightskip{0cm}\setlength\parindent{0cm}\setlength\parfillskip{0pt plus 1fil}\setlength\parskip{0.101cm}\writerlistparindent\writerlistleftskip\leavevmode\normalfont\normalsize\normalcolor\writerlistlabel\ignorespaces}{\unskip\vspace{0.21cm}\par}
\newenvironment{stylePxxiv}{\setlength\leftskip{0cm}\setlength\rightskip{0cm}\setlength\parindent{0cm}\setlength\parfillskip{0pt plus 1fil}\setlength\parskip{0.101cm}\writerlistparindent\writerlistleftskip\leavevmode\normalfont\normalsize\normalcolor\writerlistlabel\ignorespaces}{\unskip\vspace{0.21cm}\par}
\newenvironment{stylePxxv}{\setlength\leftskip{0cm}\setlength\rightskip{0cm}\setlength\parindent{0cm}\setlength\parfillskip{0pt plus 1fil}\setlength\parskip{0.101cm}\writerlistparindent\writerlistleftskip\leavevmode\normalfont\normalsize\normalcolor\writerlistlabel\ignorespaces}{\unskip\vspace{0.21cm}\par}
\newenvironment{stylePxxvi}{\setlength\leftskip{0cm}\setlength\rightskip{0cm}\setlength\parindent{0cm}\setlength\parfillskip{0pt plus 1fil}\setlength\parskip{0.101cm}\writerlistparindent\writerlistleftskip\leavevmode\normalfont\normalsize\normalcolor\writerlistlabel\ignorespaces}{\unskip\vspace{0.21cm}\par}
\newenvironment{styleTableHeading}{\setlength\leftskip{0cm plus 1fil}\setlength\rightskip{0cm plus 1fil}\setlength\parindent{0cm}\setlength\parfillskip{0pt}\setlength\parskip{0.101cm}\writerlistparindent\writerlistleftskip\leavevmode\normalfont\normalsize\normalcolor\bfseries\itshape\writerlistlabel\ignorespaces}{\unskip\vspace{0.21cm}\par}
\newenvironment{styleTableContents}{\setlength\leftskip{0cm}\setlength\rightskip{0cm}\setlength\parindent{0cm}\setlength\parfillskip{0pt plus 1fil}\setlength\parskip{0.101cm}\writerlistparindent\writerlistleftskip\leavevmode\normalfont\normalsize\normalcolor\writerlistlabel\ignorespaces}{\unskip\vspace{0.21cm}\par}
\newenvironment{stylePxxvii}{\setlength\leftskip{0cm}\setlength\rightskip{0cm}\setlength\parindent{0cm}\setlength\parfillskip{0pt plus 1fil}\setlength\parskip{0.101cm}\writerlistparindent\writerlistleftskip\leavevmode\normalfont\normalsize\normalcolor\itshape\writerlistlabel\ignorespaces}{\unskip\vspace{0.21cm}\par}
\newenvironment{stylePxxviii}{\setlength\leftskip{0cm}\setlength\rightskip{0cm}\setlength\parindent{0cm}\setlength\parfillskip{0pt plus 1fil}\setlength\parskip{0.101cm}\writerlistparindent\writerlistleftskip\leavevmode\normalfont\normalsize\normalcolor\writerlistlabel\ignorespaces}{\unskip\vspace{0.21cm}\par}
\newenvironment{stylePxxix}{\setlength\leftskip{0cm}\setlength\rightskip{0cm}\setlength\parindent{0cm}\setlength\parfillskip{0pt plus 1fil}\setlength\parskip{0.101cm}\writerlistparindent\writerlistleftskip\leavevmode\normalfont\normalsize\normalcolor\writerlistlabel\ignorespaces}{\unskip\vspace{0.21cm}\par}
\newenvironment{stylePxxx}{\setlength\leftskip{0cm}\setlength\rightskip{0cm}\setlength\parindent{0cm}\setlength\parfillskip{0pt plus 1fil}\setlength\parskip{0.101cm}\writerlistparindent\writerlistleftskip\leavevmode\normalfont\normalsize\normalcolor\writerlistlabel\ignorespaces}{\unskip\vspace{0.21cm}\par}
% List styles
\newcommand\writerlistleftskip{}
\newcommand\writerlistparindent{}
\newcommand\writerlistlabel{}
\newcommand\writerlistremovelabel{\aftergroup\let\aftergroup\writerlistparindent\aftergroup\relax\aftergroup\let\aftergroup\writerlistlabel\aftergroup\relax}
\newcommand\labellistLileveli{\textstyleBulletSymbols{{\textbullet}}}
\newcommand\labellistLilevelii{\textstyleBulletSymbols{{\textbullet}}}
\newcommand\labellistLileveliii{\textstyleBulletSymbols{{\textbullet}}}
\newcommand\labellistLileveliv{\textstyleBulletSymbols{{\textbullet}}}
\newenvironment{listLileveli}{\def\writerlistleftskip{\addtolength\leftskip{0.499cm}}\def\writerlistparindent{}\def\writerlistlabel{}\def\item{\def\writerlistparindent{\setlength\parindent{-0.499cm}}\def\writerlistlabel{\labellistLileveli\hspace{0cm}\writerlistremovelabel}}}{}
\newenvironment{listLilevelii}{\def\writerlistleftskip{\addtolength\leftskip{1.0cm}}\def\writerlistparindent{}\def\writerlistlabel{}\def\item{\def\writerlistparindent{\setlength\parindent{-0.499cm}}\def\writerlistlabel{\labellistLilevelii\hspace{0cm}\writerlistremovelabel}}}{}
\newenvironment{listLileveliii}{\def\writerlistleftskip{\addtolength\leftskip{1.4990001cm}}\def\writerlistparindent{}\def\writerlistlabel{}\def\item{\def\writerlistparindent{\setlength\parindent{-0.499cm}}\def\writerlistlabel{\labellistLileveliii\hspace{0cm}\writerlistremovelabel}}}{}
\newenvironment{listLileveliv}{\def\writerlistleftskip{\addtolength\leftskip{2.0cm}}\def\writerlistparindent{}\def\writerlistlabel{}\def\item{\def\writerlistparindent{\setlength\parindent{-0.499cm}}\def\writerlistlabel{\labellistLileveliv\hspace{0cm}\writerlistremovelabel}}}{}
\newcommand\labellistLiileveli{\textstyleBulletSymbols{{\textbullet}}}
\newcommand\labellistLiilevelii{\textstyleBulletSymbols{{\textbullet}}}
\newcommand\labellistLiileveliii{\textstyleBulletSymbols{{\textbullet}}}
\newcommand\labellistLiileveliv{\textstyleBulletSymbols{{\textbullet}}}
\newenvironment{listLiileveli}{\def\writerlistleftskip{\addtolength\leftskip{0.499cm}}\def\writerlistparindent{}\def\writerlistlabel{}\def\item{\def\writerlistparindent{\setlength\parindent{-0.499cm}}\def\writerlistlabel{\labellistLiileveli\hspace{0cm}\writerlistremovelabel}}}{}
\newenvironment{listLiilevelii}{\def\writerlistleftskip{\addtolength\leftskip{1.0cm}}\def\writerlistparindent{}\def\writerlistlabel{}\def\item{\def\writerlistparindent{\setlength\parindent{-0.499cm}}\def\writerlistlabel{\labellistLiilevelii\hspace{0cm}\writerlistremovelabel}}}{}
\newenvironment{listLiileveliii}{\def\writerlistleftskip{\addtolength\leftskip{1.4990001cm}}\def\writerlistparindent{}\def\writerlistlabel{}\def\item{\def\writerlistparindent{\setlength\parindent{-0.499cm}}\def\writerlistlabel{\labellistLiileveliii\hspace{0cm}\writerlistremovelabel}}}{}
\newenvironment{listLiileveliv}{\def\writerlistleftskip{\addtolength\leftskip{2.0cm}}\def\writerlistparindent{}\def\writerlistlabel{}\def\item{\def\writerlistparindent{\setlength\parindent{-0.499cm}}\def\writerlistlabel{\labellistLiileveliv\hspace{0cm}\writerlistremovelabel}}}{}
\newcounter{listLiiileveli}
\newcounter{listLiiilevelii}[listLiiileveli]
\newcounter{listLiiileveliii}[listLiiilevelii]
\newcounter{listLiiileveliv}[listLiiileveliii]
\renewcommand\thelistLiiileveli{\arabic{listLiiileveli}}
\renewcommand\thelistLiiilevelii{\arabic{listLiiilevelii}}
\renewcommand\thelistLiiileveliii{\arabic{listLiiileveliii}}
\renewcommand\thelistLiiileveliv{\arabic{listLiiileveliv}}
\newcommand\labellistLiiileveli{\textstyleNumberingSymbols{\thelistLiiileveli.}}
\newcommand\labellistLiiilevelii{\textstyleNumberingSymbols{\thelistLiiilevelii.}}
\newcommand\labellistLiiileveliii{\textstyleNumberingSymbols{\thelistLiiileveliii.}}
\newcommand\labellistLiiileveliv{\textstyleNumberingSymbols{\thelistLiiileveliv.}}
\newenvironment{listLiiileveli}{\def\writerlistleftskip{\addtolength\leftskip{0.499cm}}\def\writerlistparindent{}\def\writerlistlabel{}\def\item{\def\writerlistparindent{\setlength\parindent{-0.499cm}}\def\writerlistlabel{\stepcounter{listLiiileveli}\labellistLiiileveli\hspace{0cm}\writerlistremovelabel}}}{}
\newenvironment{listLiiilevelii}{\def\writerlistleftskip{\addtolength\leftskip{1.0cm}}\def\writerlistparindent{}\def\writerlistlabel{}\def\item{\def\writerlistparindent{\setlength\parindent{-0.499cm}}\def\writerlistlabel{\stepcounter{listLiiilevelii}\labellistLiiilevelii\hspace{0cm}\writerlistremovelabel}}}{}
\newenvironment{listLiiileveliii}{\def\writerlistleftskip{\addtolength\leftskip{1.4990001cm}}\def\writerlistparindent{}\def\writerlistlabel{}\def\item{\def\writerlistparindent{\setlength\parindent{-0.499cm}}\def\writerlistlabel{\stepcounter{listLiiileveliii}\labellistLiiileveliii\hspace{0cm}\writerlistremovelabel}}}{}
\newenvironment{listLiiileveliv}{\def\writerlistleftskip{\addtolength\leftskip{2.0cm}}\def\writerlistparindent{}\def\writerlistlabel{}\def\item{\def\writerlistparindent{\setlength\parindent{-0.499cm}}\def\writerlistlabel{\stepcounter{listLiiileveliv}\labellistLiiileveliv\hspace{0cm}\writerlistremovelabel}}}{}
\newcounter{listLivleveli}
\newcounter{listLivlevelii}[listLivleveli]
\newcounter{listLivleveliii}[listLivlevelii]
\newcounter{listLivleveliv}[listLivleveliii]
\renewcommand\thelistLivleveli{\arabic{listLivleveli}}
\renewcommand\thelistLivlevelii{\arabic{listLivlevelii}}
\renewcommand\thelistLivleveliii{\arabic{listLivleveliii}}
\renewcommand\thelistLivleveliv{\arabic{listLivleveliv}}
\newcommand\labellistLivleveli{\textstyleNumberingSymbols{\thelistLivleveli.}}
\newcommand\labellistLivlevelii{\textstyleNumberingSymbols{\thelistLivlevelii.}}
\newcommand\labellistLivleveliii{\textstyleNumberingSymbols{\thelistLivleveliii.}}
\newcommand\labellistLivleveliv{\textstyleNumberingSymbols{\thelistLivleveliv.}}
\newenvironment{listLivleveli}{\def\writerlistleftskip{\addtolength\leftskip{0.499cm}}\def\writerlistparindent{}\def\writerlistlabel{}\def\item{\def\writerlistparindent{\setlength\parindent{-0.499cm}}\def\writerlistlabel{\stepcounter{listLivleveli}\labellistLivleveli\hspace{0cm}\writerlistremovelabel}}}{}
\newenvironment{listLivlevelii}{\def\writerlistleftskip{\addtolength\leftskip{1.0cm}}\def\writerlistparindent{}\def\writerlistlabel{}\def\item{\def\writerlistparindent{\setlength\parindent{-0.499cm}}\def\writerlistlabel{\stepcounter{listLivlevelii}\labellistLivlevelii\hspace{0cm}\writerlistremovelabel}}}{}
\newenvironment{listLivleveliii}{\def\writerlistleftskip{\addtolength\leftskip{1.4990001cm}}\def\writerlistparindent{}\def\writerlistlabel{}\def\item{\def\writerlistparindent{\setlength\parindent{-0.499cm}}\def\writerlistlabel{\stepcounter{listLivleveliii}\labellistLivleveliii\hspace{0cm}\writerlistremovelabel}}}{}
\newenvironment{listLivleveliv}{\def\writerlistleftskip{\addtolength\leftskip{2.0cm}}\def\writerlistparindent{}\def\writerlistlabel{}\def\item{\def\writerlistparindent{\setlength\parindent{-0.499cm}}\def\writerlistlabel{\stepcounter{listLivleveliv}\labellistLivleveliv\hspace{0cm}\writerlistremovelabel}}}{}
\newcommand\labellistLvleveli{\textstyleBulletSymbols{{\textbullet}}}
\newcommand\labellistLvlevelii{\textstyleBulletSymbols{{\textbullet}}}
\newcommand\labellistLvleveliii{\textstyleBulletSymbols{{\textbullet}}}
\newcommand\labellistLvleveliv{\textstyleBulletSymbols{{\textbullet}}}
\newenvironment{listLvleveli}{\def\writerlistleftskip{\addtolength\leftskip{0.499cm}}\def\writerlistparindent{}\def\writerlistlabel{}\def\item{\def\writerlistparindent{\setlength\parindent{-0.499cm}}\def\writerlistlabel{\labellistLvleveli\hspace{0cm}\writerlistremovelabel}}}{}
\newenvironment{listLvlevelii}{\def\writerlistleftskip{\addtolength\leftskip{1.0cm}}\def\writerlistparindent{}\def\writerlistlabel{}\def\item{\def\writerlistparindent{\setlength\parindent{-0.499cm}}\def\writerlistlabel{\labellistLvlevelii\hspace{0cm}\writerlistremovelabel}}}{}
\newenvironment{listLvleveliii}{\def\writerlistleftskip{\addtolength\leftskip{1.4990001cm}}\def\writerlistparindent{}\def\writerlistlabel{}\def\item{\def\writerlistparindent{\setlength\parindent{-0.499cm}}\def\writerlistlabel{\labellistLvleveliii\hspace{0cm}\writerlistremovelabel}}}{}
\newenvironment{listLvleveliv}{\def\writerlistleftskip{\addtolength\leftskip{2.0cm}}\def\writerlistparindent{}\def\writerlistlabel{}\def\item{\def\writerlistparindent{\setlength\parindent{-0.499cm}}\def\writerlistlabel{\labellistLvleveliv\hspace{0cm}\writerlistremovelabel}}}{}
\newcommand\labellistLvileveli{\textstyleBulletSymbols{{\textbullet}}}
\newcommand\labellistLvilevelii{\textstyleBulletSymbols{{\textbullet}}}
\newcommand\labellistLvileveliii{\textstyleBulletSymbols{{\textbullet}}}
\newcommand\labellistLvileveliv{\textstyleBulletSymbols{{\textbullet}}}
\newenvironment{listLvileveli}{\def\writerlistleftskip{\addtolength\leftskip{0.499cm}}\def\writerlistparindent{}\def\writerlistlabel{}\def\item{\def\writerlistparindent{\setlength\parindent{-0.499cm}}\def\writerlistlabel{\labellistLvileveli\hspace{0cm}\writerlistremovelabel}}}{}
\newenvironment{listLvilevelii}{\def\writerlistleftskip{\addtolength\leftskip{1.0cm}}\def\writerlistparindent{}\def\writerlistlabel{}\def\item{\def\writerlistparindent{\setlength\parindent{-0.499cm}}\def\writerlistlabel{\labellistLvilevelii\hspace{0cm}\writerlistremovelabel}}}{}
\newenvironment{listLvileveliii}{\def\writerlistleftskip{\addtolength\leftskip{1.4990001cm}}\def\writerlistparindent{}\def\writerlistlabel{}\def\item{\def\writerlistparindent{\setlength\parindent{-0.499cm}}\def\writerlistlabel{\labellistLvileveliii\hspace{0cm}\writerlistremovelabel}}}{}
\newenvironment{listLvileveliv}{\def\writerlistleftskip{\addtolength\leftskip{2.0cm}}\def\writerlistparindent{}\def\writerlistlabel{}\def\item{\def\writerlistparindent{\setlength\parindent{-0.499cm}}\def\writerlistlabel{\labellistLvileveliv\hspace{0cm}\writerlistremovelabel}}}{}
\newcounter{listLviileveli}
\newcounter{listLviilevelii}[listLviileveli]
\newcounter{listLviileveliii}[listLviilevelii]
\newcounter{listLviileveliv}[listLviileveliii]
\renewcommand\thelistLviileveli{\arabic{listLviileveli}}
\renewcommand\thelistLviilevelii{\arabic{listLviilevelii}}
\renewcommand\thelistLviileveliii{\arabic{listLviileveliii}}
\renewcommand\thelistLviileveliv{\arabic{listLviileveliv}}
\newcommand\labellistLviileveli{\textstyleNumberingSymbols{\thelistLviileveli.}}
\newcommand\labellistLviilevelii{\textstyleNumberingSymbols{\thelistLviilevelii.}}
\newcommand\labellistLviileveliii{\textstyleNumberingSymbols{\thelistLviileveliii.}}
\newcommand\labellistLviileveliv{\textstyleNumberingSymbols{\thelistLviileveliv.}}
\newenvironment{listLviileveli}{\def\writerlistleftskip{\addtolength\leftskip{0.499cm}}\def\writerlistparindent{}\def\writerlistlabel{}\def\item{\def\writerlistparindent{\setlength\parindent{-0.499cm}}\def\writerlistlabel{\stepcounter{listLviileveli}\labellistLviileveli\hspace{0cm}\writerlistremovelabel}}}{}
\newenvironment{listLviilevelii}{\def\writerlistleftskip{\addtolength\leftskip{1.0cm}}\def\writerlistparindent{}\def\writerlistlabel{}\def\item{\def\writerlistparindent{\setlength\parindent{-0.499cm}}\def\writerlistlabel{\stepcounter{listLviilevelii}\labellistLviilevelii\hspace{0cm}\writerlistremovelabel}}}{}
\newenvironment{listLviileveliii}{\def\writerlistleftskip{\addtolength\leftskip{1.4990001cm}}\def\writerlistparindent{}\def\writerlistlabel{}\def\item{\def\writerlistparindent{\setlength\parindent{-0.499cm}}\def\writerlistlabel{\stepcounter{listLviileveliii}\labellistLviileveliii\hspace{0cm}\writerlistremovelabel}}}{}
\newenvironment{listLviileveliv}{\def\writerlistleftskip{\addtolength\leftskip{2.0cm}}\def\writerlistparindent{}\def\writerlistlabel{}\def\item{\def\writerlistparindent{\setlength\parindent{-0.499cm}}\def\writerlistlabel{\stepcounter{listLviileveliv}\labellistLviileveliv\hspace{0cm}\writerlistremovelabel}}}{}
\newcounter{listLviiileveli}
\newcounter{listLviiilevelii}[listLviiileveli]
\newcounter{listLviiileveliii}[listLviiilevelii]
\newcounter{listLviiileveliv}[listLviiileveliii]
\renewcommand\thelistLviiileveli{\arabic{listLviiileveli}}
\renewcommand\thelistLviiilevelii{\arabic{listLviiilevelii}}
\renewcommand\thelistLviiileveliii{\arabic{listLviiileveliii}}
\renewcommand\thelistLviiileveliv{\arabic{listLviiileveliv}}
\newcommand\labellistLviiileveli{\textstyleNumberingSymbols{\thelistLviiileveli.}}
\newcommand\labellistLviiilevelii{\textstyleNumberingSymbols{\thelistLviiilevelii.}}
\newcommand\labellistLviiileveliii{\textstyleNumberingSymbols{\thelistLviiileveliii.}}
\newcommand\labellistLviiileveliv{\textstyleNumberingSymbols{\thelistLviiileveliv.}}
\newenvironment{listLviiileveli}{\def\writerlistleftskip{\addtolength\leftskip{0.499cm}}\def\writerlistparindent{}\def\writerlistlabel{}\def\item{\def\writerlistparindent{\setlength\parindent{-0.499cm}}\def\writerlistlabel{\stepcounter{listLviiileveli}\labellistLviiileveli\hspace{0cm}\writerlistremovelabel}}}{}
\newenvironment{listLviiilevelii}{\def\writerlistleftskip{\addtolength\leftskip{1.0cm}}\def\writerlistparindent{}\def\writerlistlabel{}\def\item{\def\writerlistparindent{\setlength\parindent{-0.499cm}}\def\writerlistlabel{\stepcounter{listLviiilevelii}\labellistLviiilevelii\hspace{0cm}\writerlistremovelabel}}}{}
\newenvironment{listLviiileveliii}{\def\writerlistleftskip{\addtolength\leftskip{1.4990001cm}}\def\writerlistparindent{}\def\writerlistlabel{}\def\item{\def\writerlistparindent{\setlength\parindent{-0.499cm}}\def\writerlistlabel{\stepcounter{listLviiileveliii}\labellistLviiileveliii\hspace{0cm}\writerlistremovelabel}}}{}
\newenvironment{listLviiileveliv}{\def\writerlistleftskip{\addtolength\leftskip{2.0cm}}\def\writerlistparindent{}\def\writerlistlabel{}\def\item{\def\writerlistparindent{\setlength\parindent{-0.499cm}}\def\writerlistlabel{\stepcounter{listLviiileveliv}\labellistLviiileveliv\hspace{0cm}\writerlistremovelabel}}}{}
\newcommand\labellistLixleveli{\textstyleBulletSymbols{{\textbullet}}}
\newcommand\labellistLixlevelii{\textstyleBulletSymbols{{\textbullet}}}
\newcommand\labellistLixleveliii{\textstyleBulletSymbols{{\textbullet}}}
\newcommand\labellistLixleveliv{\textstyleBulletSymbols{{\textbullet}}}
\newenvironment{listLixleveli}{\def\writerlistleftskip{\addtolength\leftskip{0.499cm}}\def\writerlistparindent{}\def\writerlistlabel{}\def\item{\def\writerlistparindent{\setlength\parindent{-0.499cm}}\def\writerlistlabel{\labellistLixleveli\hspace{0cm}\writerlistremovelabel}}}{}
\newenvironment{listLixlevelii}{\def\writerlistleftskip{\addtolength\leftskip{1.0cm}}\def\writerlistparindent{}\def\writerlistlabel{}\def\item{\def\writerlistparindent{\setlength\parindent{-0.499cm}}\def\writerlistlabel{\labellistLixlevelii\hspace{0cm}\writerlistremovelabel}}}{}
\newenvironment{listLixleveliii}{\def\writerlistleftskip{\addtolength\leftskip{1.4990001cm}}\def\writerlistparindent{}\def\writerlistlabel{}\def\item{\def\writerlistparindent{\setlength\parindent{-0.499cm}}\def\writerlistlabel{\labellistLixleveliii\hspace{0cm}\writerlistremovelabel}}}{}
\newenvironment{listLixleveliv}{\def\writerlistleftskip{\addtolength\leftskip{2.0cm}}\def\writerlistparindent{}\def\writerlistlabel{}\def\item{\def\writerlistparindent{\setlength\parindent{-0.499cm}}\def\writerlistlabel{\labellistLixleveliv\hspace{0cm}\writerlistremovelabel}}}{}
\newcommand\labellistLxleveli{\textstyleBulletSymbols{{\textbullet}}}
\newcommand\labellistLxlevelii{\textstyleBulletSymbols{{\textbullet}}}
\newcommand\labellistLxleveliii{\textstyleBulletSymbols{{\textbullet}}}
\newcommand\labellistLxleveliv{\textstyleBulletSymbols{{\textbullet}}}
\newenvironment{listLxleveli}{\def\writerlistleftskip{\addtolength\leftskip{0.499cm}}\def\writerlistparindent{}\def\writerlistlabel{}\def\item{\def\writerlistparindent{\setlength\parindent{-0.499cm}}\def\writerlistlabel{\labellistLxleveli\hspace{0cm}\writerlistremovelabel}}}{}
\newenvironment{listLxlevelii}{\def\writerlistleftskip{\addtolength\leftskip{1.0cm}}\def\writerlistparindent{}\def\writerlistlabel{}\def\item{\def\writerlistparindent{\setlength\parindent{-0.499cm}}\def\writerlistlabel{\labellistLxlevelii\hspace{0cm}\writerlistremovelabel}}}{}
\newenvironment{listLxleveliii}{\def\writerlistleftskip{\addtolength\leftskip{1.4990001cm}}\def\writerlistparindent{}\def\writerlistlabel{}\def\item{\def\writerlistparindent{\setlength\parindent{-0.499cm}}\def\writerlistlabel{\labellistLxleveliii\hspace{0cm}\writerlistremovelabel}}}{}
\newenvironment{listLxleveliv}{\def\writerlistleftskip{\addtolength\leftskip{2.0cm}}\def\writerlistparindent{}\def\writerlistlabel{}\def\item{\def\writerlistparindent{\setlength\parindent{-0.499cm}}\def\writerlistlabel{\labellistLxleveliv\hspace{0cm}\writerlistremovelabel}}}{}
\newcommand\labellistLxileveli{\textstyleBulletSymbols{{\textbullet}}}
\newcommand\labellistLxilevelii{\textstyleBulletSymbols{{\textbullet}}}
\newcommand\labellistLxileveliii{\textstyleBulletSymbols{{\textbullet}}}
\newcommand\labellistLxileveliv{\textstyleBulletSymbols{{\textbullet}}}
\newenvironment{listLxileveli}{\def\writerlistleftskip{\addtolength\leftskip{0.499cm}}\def\writerlistparindent{}\def\writerlistlabel{}\def\item{\def\writerlistparindent{\setlength\parindent{-0.499cm}}\def\writerlistlabel{\labellistLxileveli\hspace{0cm}\writerlistremovelabel}}}{}
\newenvironment{listLxilevelii}{\def\writerlistleftskip{\addtolength\leftskip{1.0cm}}\def\writerlistparindent{}\def\writerlistlabel{}\def\item{\def\writerlistparindent{\setlength\parindent{-0.499cm}}\def\writerlistlabel{\labellistLxilevelii\hspace{0cm}\writerlistremovelabel}}}{}
\newenvironment{listLxileveliii}{\def\writerlistleftskip{\addtolength\leftskip{1.4990001cm}}\def\writerlistparindent{}\def\writerlistlabel{}\def\item{\def\writerlistparindent{\setlength\parindent{-0.499cm}}\def\writerlistlabel{\labellistLxileveliii\hspace{0cm}\writerlistremovelabel}}}{}
\newenvironment{listLxileveliv}{\def\writerlistleftskip{\addtolength\leftskip{2.0cm}}\def\writerlistparindent{}\def\writerlistlabel{}\def\item{\def\writerlistparindent{\setlength\parindent{-0.499cm}}\def\writerlistlabel{\labellistLxileveliv\hspace{0cm}\writerlistremovelabel}}}{}
\newcommand\labellistLxiileveli{\textstyleBulletSymbols{{\textbullet}}}
\newcommand\labellistLxiilevelii{\textstyleBulletSymbols{{\textbullet}}}
\newcommand\labellistLxiileveliii{\textstyleBulletSymbols{{\textbullet}}}
\newcommand\labellistLxiileveliv{\textstyleBulletSymbols{{\textbullet}}}
\newenvironment{listLxiileveli}{\def\writerlistleftskip{\addtolength\leftskip{0.499cm}}\def\writerlistparindent{}\def\writerlistlabel{}\def\item{\def\writerlistparindent{\setlength\parindent{-0.499cm}}\def\writerlistlabel{\labellistLxiileveli\hspace{0cm}\writerlistremovelabel}}}{}
\newenvironment{listLxiilevelii}{\def\writerlistleftskip{\addtolength\leftskip{1.0cm}}\def\writerlistparindent{}\def\writerlistlabel{}\def\item{\def\writerlistparindent{\setlength\parindent{-0.499cm}}\def\writerlistlabel{\labellistLxiilevelii\hspace{0cm}\writerlistremovelabel}}}{}
\newenvironment{listLxiileveliii}{\def\writerlistleftskip{\addtolength\leftskip{1.4990001cm}}\def\writerlistparindent{}\def\writerlistlabel{}\def\item{\def\writerlistparindent{\setlength\parindent{-0.499cm}}\def\writerlistlabel{\labellistLxiileveliii\hspace{0cm}\writerlistremovelabel}}}{}
\newenvironment{listLxiileveliv}{\def\writerlistleftskip{\addtolength\leftskip{2.0cm}}\def\writerlistparindent{}\def\writerlistlabel{}\def\item{\def\writerlistparindent{\setlength\parindent{-0.499cm}}\def\writerlistlabel{\labellistLxiileveliv\hspace{0cm}\writerlistremovelabel}}}{}
\newcommand\labellistLxiiileveli{\textstyleBulletSymbols{{\textbullet}}}
\newcommand\labellistLxiiilevelii{\textstyleBulletSymbols{{\textbullet}}}
\newcommand\labellistLxiiileveliii{\textstyleBulletSymbols{{\textbullet}}}
\newcommand\labellistLxiiileveliv{\textstyleBulletSymbols{{\textbullet}}}
\newenvironment{listLxiiileveli}{\def\writerlistleftskip{\addtolength\leftskip{0.499cm}}\def\writerlistparindent{}\def\writerlistlabel{}\def\item{\def\writerlistparindent{\setlength\parindent{-0.499cm}}\def\writerlistlabel{\labellistLxiiileveli\hspace{0cm}\writerlistremovelabel}}}{}
\newenvironment{listLxiiilevelii}{\def\writerlistleftskip{\addtolength\leftskip{1.0cm}}\def\writerlistparindent{}\def\writerlistlabel{}\def\item{\def\writerlistparindent{\setlength\parindent{-0.499cm}}\def\writerlistlabel{\labellistLxiiilevelii\hspace{0cm}\writerlistremovelabel}}}{}
\newenvironment{listLxiiileveliii}{\def\writerlistleftskip{\addtolength\leftskip{1.4990001cm}}\def\writerlistparindent{}\def\writerlistlabel{}\def\item{\def\writerlistparindent{\setlength\parindent{-0.499cm}}\def\writerlistlabel{\labellistLxiiileveliii\hspace{0cm}\writerlistremovelabel}}}{}
\newenvironment{listLxiiileveliv}{\def\writerlistleftskip{\addtolength\leftskip{2.0cm}}\def\writerlistparindent{}\def\writerlistlabel{}\def\item{\def\writerlistparindent{\setlength\parindent{-0.499cm}}\def\writerlistlabel{\labellistLxiiileveliv\hspace{0cm}\writerlistremovelabel}}}{}
\newcommand\labellistLxivleveli{\textstyleBulletSymbols{{\textbullet}}}
\newcommand\labellistLxivlevelii{\textstyleBulletSymbols{{\textbullet}}}
\newcommand\labellistLxivleveliii{\textstyleBulletSymbols{{\textbullet}}}
\newcommand\labellistLxivleveliv{\textstyleBulletSymbols{{\textbullet}}}
\newenvironment{listLxivleveli}{\def\writerlistleftskip{\addtolength\leftskip{0.499cm}}\def\writerlistparindent{}\def\writerlistlabel{}\def\item{\def\writerlistparindent{\setlength\parindent{-0.499cm}}\def\writerlistlabel{\labellistLxivleveli\hspace{0cm}\writerlistremovelabel}}}{}
\newenvironment{listLxivlevelii}{\def\writerlistleftskip{\addtolength\leftskip{1.0cm}}\def\writerlistparindent{}\def\writerlistlabel{}\def\item{\def\writerlistparindent{\setlength\parindent{-0.499cm}}\def\writerlistlabel{\labellistLxivlevelii\hspace{0cm}\writerlistremovelabel}}}{}
\newenvironment{listLxivleveliii}{\def\writerlistleftskip{\addtolength\leftskip{1.4990001cm}}\def\writerlistparindent{}\def\writerlistlabel{}\def\item{\def\writerlistparindent{\setlength\parindent{-0.499cm}}\def\writerlistlabel{\labellistLxivleveliii\hspace{0cm}\writerlistremovelabel}}}{}
\newenvironment{listLxivleveliv}{\def\writerlistleftskip{\addtolength\leftskip{2.0cm}}\def\writerlistparindent{}\def\writerlistlabel{}\def\item{\def\writerlistparindent{\setlength\parindent{-0.499cm}}\def\writerlistlabel{\labellistLxivleveliv\hspace{0cm}\writerlistremovelabel}}}{}
\newcommand\labellistLxvleveli{\textstyleBulletSymbols{{\textbullet}}}
\newcommand\labellistLxvlevelii{\textstyleBulletSymbols{{\textbullet}}}
\newcommand\labellistLxvleveliii{\textstyleBulletSymbols{{\textbullet}}}
\newcommand\labellistLxvleveliv{\textstyleBulletSymbols{{\textbullet}}}
\newenvironment{listLxvleveli}{\def\writerlistleftskip{\addtolength\leftskip{0.499cm}}\def\writerlistparindent{}\def\writerlistlabel{}\def\item{\def\writerlistparindent{\setlength\parindent{-0.499cm}}\def\writerlistlabel{\labellistLxvleveli\hspace{0cm}\writerlistremovelabel}}}{}
\newenvironment{listLxvlevelii}{\def\writerlistleftskip{\addtolength\leftskip{1.0cm}}\def\writerlistparindent{}\def\writerlistlabel{}\def\item{\def\writerlistparindent{\setlength\parindent{-0.499cm}}\def\writerlistlabel{\labellistLxvlevelii\hspace{0cm}\writerlistremovelabel}}}{}
\newenvironment{listLxvleveliii}{\def\writerlistleftskip{\addtolength\leftskip{1.4990001cm}}\def\writerlistparindent{}\def\writerlistlabel{}\def\item{\def\writerlistparindent{\setlength\parindent{-0.499cm}}\def\writerlistlabel{\labellistLxvleveliii\hspace{0cm}\writerlistremovelabel}}}{}
\newenvironment{listLxvleveliv}{\def\writerlistleftskip{\addtolength\leftskip{2.0cm}}\def\writerlistparindent{}\def\writerlistlabel{}\def\item{\def\writerlistparindent{\setlength\parindent{-0.499cm}}\def\writerlistlabel{\labellistLxvleveliv\hspace{0cm}\writerlistremovelabel}}}{}
\newcommand\labellistLxvileveli{\textstyleBulletSymbols{{\textbullet}}}
\newcommand\labellistLxvilevelii{\textstyleBulletSymbols{{\textbullet}}}
\newcommand\labellistLxvileveliii{\textstyleBulletSymbols{{\textbullet}}}
\newcommand\labellistLxvileveliv{\textstyleBulletSymbols{{\textbullet}}}
\newenvironment{listLxvileveli}{\def\writerlistleftskip{\addtolength\leftskip{0.499cm}}\def\writerlistparindent{}\def\writerlistlabel{}\def\item{\def\writerlistparindent{\setlength\parindent{-0.499cm}}\def\writerlistlabel{\labellistLxvileveli\hspace{0cm}\writerlistremovelabel}}}{}
\newenvironment{listLxvilevelii}{\def\writerlistleftskip{\addtolength\leftskip{1.0cm}}\def\writerlistparindent{}\def\writerlistlabel{}\def\item{\def\writerlistparindent{\setlength\parindent{-0.499cm}}\def\writerlistlabel{\labellistLxvilevelii\hspace{0cm}\writerlistremovelabel}}}{}
\newenvironment{listLxvileveliii}{\def\writerlistleftskip{\addtolength\leftskip{1.4990001cm}}\def\writerlistparindent{}\def\writerlistlabel{}\def\item{\def\writerlistparindent{\setlength\parindent{-0.499cm}}\def\writerlistlabel{\labellistLxvileveliii\hspace{0cm}\writerlistremovelabel}}}{}
\newenvironment{listLxvileveliv}{\def\writerlistleftskip{\addtolength\leftskip{2.0cm}}\def\writerlistparindent{}\def\writerlistlabel{}\def\item{\def\writerlistparindent{\setlength\parindent{-0.499cm}}\def\writerlistlabel{\labellistLxvileveliv\hspace{0cm}\writerlistremovelabel}}}{}
\newcommand\labellistLxviileveli{\textstyleBulletSymbols{{\textbullet}}}
\newcommand\labellistLxviilevelii{\textstyleBulletSymbols{{\textbullet}}}
\newcommand\labellistLxviileveliii{\textstyleBulletSymbols{{\textbullet}}}
\newcommand\labellistLxviileveliv{\textstyleBulletSymbols{{\textbullet}}}
\newenvironment{listLxviileveli}{\def\writerlistleftskip{\addtolength\leftskip{0.499cm}}\def\writerlistparindent{}\def\writerlistlabel{}\def\item{\def\writerlistparindent{\setlength\parindent{-0.499cm}}\def\writerlistlabel{\labellistLxviileveli\hspace{0cm}\writerlistremovelabel}}}{}
\newenvironment{listLxviilevelii}{\def\writerlistleftskip{\addtolength\leftskip{1.0cm}}\def\writerlistparindent{}\def\writerlistlabel{}\def\item{\def\writerlistparindent{\setlength\parindent{-0.499cm}}\def\writerlistlabel{\labellistLxviilevelii\hspace{0cm}\writerlistremovelabel}}}{}
\newenvironment{listLxviileveliii}{\def\writerlistleftskip{\addtolength\leftskip{1.4990001cm}}\def\writerlistparindent{}\def\writerlistlabel{}\def\item{\def\writerlistparindent{\setlength\parindent{-0.499cm}}\def\writerlistlabel{\labellistLxviileveliii\hspace{0cm}\writerlistremovelabel}}}{}
\newenvironment{listLxviileveliv}{\def\writerlistleftskip{\addtolength\leftskip{2.0cm}}\def\writerlistparindent{}\def\writerlistlabel{}\def\item{\def\writerlistparindent{\setlength\parindent{-0.499cm}}\def\writerlistlabel{\labellistLxviileveliv\hspace{0cm}\writerlistremovelabel}}}{}
\newcommand\labellistLxviiileveli{\textstyleBulletSymbols{{\textbullet}}}
\newcommand\labellistLxviiilevelii{\textstyleBulletSymbols{{\textbullet}}}
\newcommand\labellistLxviiileveliii{\textstyleBulletSymbols{{\textbullet}}}
\newcommand\labellistLxviiileveliv{\textstyleBulletSymbols{{\textbullet}}}
\newenvironment{listLxviiileveli}{\def\writerlistleftskip{\addtolength\leftskip{0.499cm}}\def\writerlistparindent{}\def\writerlistlabel{}\def\item{\def\writerlistparindent{\setlength\parindent{-0.499cm}}\def\writerlistlabel{\labellistLxviiileveli\hspace{0cm}\writerlistremovelabel}}}{}
\newenvironment{listLxviiilevelii}{\def\writerlistleftskip{\addtolength\leftskip{1.0cm}}\def\writerlistparindent{}\def\writerlistlabel{}\def\item{\def\writerlistparindent{\setlength\parindent{-0.499cm}}\def\writerlistlabel{\labellistLxviiilevelii\hspace{0cm}\writerlistremovelabel}}}{}
\newenvironment{listLxviiileveliii}{\def\writerlistleftskip{\addtolength\leftskip{1.4990001cm}}\def\writerlistparindent{}\def\writerlistlabel{}\def\item{\def\writerlistparindent{\setlength\parindent{-0.499cm}}\def\writerlistlabel{\labellistLxviiileveliii\hspace{0cm}\writerlistremovelabel}}}{}
\newenvironment{listLxviiileveliv}{\def\writerlistleftskip{\addtolength\leftskip{2.0cm}}\def\writerlistparindent{}\def\writerlistlabel{}\def\item{\def\writerlistparindent{\setlength\parindent{-0.499cm}}\def\writerlistlabel{\labellistLxviiileveliv\hspace{0cm}\writerlistremovelabel}}}{}
\newcommand\labellistLxixleveli{\textstyleBulletSymbols{{\textbullet}}}
\newcommand\labellistLxixlevelii{\textstyleBulletSymbols{{\textbullet}}}
\newcommand\labellistLxixleveliii{\textstyleBulletSymbols{{\textbullet}}}
\newcommand\labellistLxixleveliv{\textstyleBulletSymbols{{\textbullet}}}
\newenvironment{listLxixleveli}{\def\writerlistleftskip{\addtolength\leftskip{0.499cm}}\def\writerlistparindent{}\def\writerlistlabel{}\def\item{\def\writerlistparindent{\setlength\parindent{-0.499cm}}\def\writerlistlabel{\labellistLxixleveli\hspace{0cm}\writerlistremovelabel}}}{}
\newenvironment{listLxixlevelii}{\def\writerlistleftskip{\addtolength\leftskip{1.0cm}}\def\writerlistparindent{}\def\writerlistlabel{}\def\item{\def\writerlistparindent{\setlength\parindent{-0.499cm}}\def\writerlistlabel{\labellistLxixlevelii\hspace{0cm}\writerlistremovelabel}}}{}
\newenvironment{listLxixleveliii}{\def\writerlistleftskip{\addtolength\leftskip{1.4990001cm}}\def\writerlistparindent{}\def\writerlistlabel{}\def\item{\def\writerlistparindent{\setlength\parindent{-0.499cm}}\def\writerlistlabel{\labellistLxixleveliii\hspace{0cm}\writerlistremovelabel}}}{}
\newenvironment{listLxixleveliv}{\def\writerlistleftskip{\addtolength\leftskip{2.0cm}}\def\writerlistparindent{}\def\writerlistlabel{}\def\item{\def\writerlistparindent{\setlength\parindent{-0.499cm}}\def\writerlistlabel{\labellistLxixleveliv\hspace{0cm}\writerlistremovelabel}}}{}
\newcommand\labellistLxxleveli{\textstyleBulletSymbols{{\textbullet}}}
\newcommand\labellistLxxlevelii{\textstyleBulletSymbols{{\textbullet}}}
\newcommand\labellistLxxleveliii{\textstyleBulletSymbols{{\textbullet}}}
\newcommand\labellistLxxleveliv{\textstyleBulletSymbols{{\textbullet}}}
\newenvironment{listLxxleveli}{\def\writerlistleftskip{\addtolength\leftskip{0.499cm}}\def\writerlistparindent{}\def\writerlistlabel{}\def\item{\def\writerlistparindent{\setlength\parindent{-0.499cm}}\def\writerlistlabel{\labellistLxxleveli\hspace{0cm}\writerlistremovelabel}}}{}
\newenvironment{listLxxlevelii}{\def\writerlistleftskip{\addtolength\leftskip{1.0cm}}\def\writerlistparindent{}\def\writerlistlabel{}\def\item{\def\writerlistparindent{\setlength\parindent{-0.499cm}}\def\writerlistlabel{\labellistLxxlevelii\hspace{0cm}\writerlistremovelabel}}}{}
\newenvironment{listLxxleveliii}{\def\writerlistleftskip{\addtolength\leftskip{1.4990001cm}}\def\writerlistparindent{}\def\writerlistlabel{}\def\item{\def\writerlistparindent{\setlength\parindent{-0.499cm}}\def\writerlistlabel{\labellistLxxleveliii\hspace{0cm}\writerlistremovelabel}}}{}
\newenvironment{listLxxleveliv}{\def\writerlistleftskip{\addtolength\leftskip{2.0cm}}\def\writerlistparindent{}\def\writerlistlabel{}\def\item{\def\writerlistparindent{\setlength\parindent{-0.499cm}}\def\writerlistlabel{\labellistLxxleveliv\hspace{0cm}\writerlistremovelabel}}}{}
\setlength\tabcolsep{1mm}
\renewcommand\arraystretch{1.3}
% footnotes configuration
\makeatletter
\renewcommand\thefootnote{\arabic{footnote}}
\renewcommand\@makefnmark{\mbox{\textstyleFootnoteanchor{\@thefnmark}}}
\makeatother
\usepackage{palatino}
\usepackage[bluelace,screen,nopanel,sectionbreak]{pdfscreen}
%\hypersetup{pdfpagemode={FullScreen}}
\margins{0.5in}{0.5in}{0.5in}{0.5in}
\screensize{6in}{8in}
\sloppy
\begin{document}
\begin{stylePi}
User{\textquotesingle}s manual for
\end{stylePi}

\begin{stylePii}
 \includegraphics[width=10.98cm,height=2.064cm]{w2llogo.png} 
\end{stylePii}

\begin{stylePiii}
Writer2LaTeX, Writer2BibTeX, Writer2xhtml and Calc2xhtml
\end{stylePiii}

\begin{stylePiv}
version 0.4
\end{stylePiv}

\begin{stylePv}
{\textcopyright} 2002{--}2005 Henrik Just
\end{stylePv}

\setcounter{tocdepth}{1}
\renewcommand\contentsname{Table of Contents}
\tableofcontents
\section{Introduction}
\subsection{What is Writer2LaTeX?}
\begin{styleTextbody}
Writer2LaTeX is a utility to convert OpenOffice.org (or StarOffice 6/7)
Writer and Calc documents\footnote{The new OASIS OpenDocument format is
not yet supported. This is planned for the next version of
Writer2LaTeX.} {--} in particular documents containing formulas {--}
into other formats, Actually it is 4 converters in one:
\end{styleTextbody}

\begin{listLileveli}
\item 
\begin{stylePvii}
\textbf{Writer2LaTeX} converts Writer documents to LaTeX 2e.
\end{stylePvii}
\item 
\begin{stylePvii}
\textbf{Writer2BibTeX} extracts bibliographic data from a Writer
document and converts it to BibTeX format.
\end{stylePvii}
\item 
\begin{stylePvii}
\textbf{Writer2xhtml} converts Writer documents to XHTML 1.0 strict or
XHTML 1.1 + MathML 2.0, using CSS2 to convert style information.
\end{stylePvii}
\item 
\begin{stylePvii}
\textbf{Calc2xhtml} converts Calc documents to XHTML 1.0 strict, using
CSS2 to convert style information.
\end{stylePvii}
\end{listLileveli}
\begin{styleTextbody}
You can use Writer2LaTeX
\end{styleTextbody}

\begin{listLiileveli}
\item 
\begin{stylePviii}
...as a command line utility, independent of OpenOffice.org/StarOffice. 
\end{stylePviii}
\item 
\begin{stylePviii}
...as an export filter for OpenOffice.org 1.1 or StarOffice 7.
\end{stylePviii}
\item 
\begin{stylePviii}
...from another Java program.
\end{stylePviii}
\end{listLiileveli}
\begin{styleTextbody}
This user{\textquotesingle}s manual will explain how to install and use
Writer2LaTeX.
\end{styleTextbody}

\begin{styleTextbody}
Writer2LaTeX is a Java application, and thus should work on any platform
that supports Java. You need Sun{\textquotesingle}s Java 2 Virtual
Machine (Runtime Environment), \textbf{version 1.4} or \textbf{1.5}.
You can download this from
\textstyleTeletype{http://java.sun.com/getjava/download.html}. AFAIK
Writer2LaTeX doesn{\textquotesingle}t run (unmodified) under any other
Java interpreter.
\end{styleTextbody}

\begin{styleTextbody}
\textit{Note}: In this manual OOo is used as an abbreviation of
OpenOffice.org/StarOffice.
\end{styleTextbody}

\section{Installation}
\subsection{How to install Writer2LaTeX for command line usage}
\begin{styleTextbody}
Writer2LaTeX can work as a standalone command line utility (that is
without OOo).
\end{styleTextbody}

\subsubsection{Installation for Microsoft Windows}
\begin{styleTextbody}
To install Writer2LaTeX under Microsoft Windows follow these
instructions:
\end{styleTextbody}

\setcounter{listLiiileveli}{0}
\begin{listLiiileveli}
\item 
\begin{stylePix}
Unzip \textstyleSourceText{writer2latex04.zip} into some directory. This
will create a subdirectory \textstyleSourceText{writer2latex04}.
\end{stylePix}
\item 
\begin{stylePix}
Add this directory to your PATH environment variable.
\end{stylePix}
\item 
\begin{stylePix}
Open the file \textstyleSourceText{w2l.bat} with a text editor and
replace the path at the top of the file with the full path to
Writer2LaTeX, for example
\end{stylePix}

\begin{stylePx}
\textstyleSourceText{set
W2LPATH={\textquotedbl}c:{\textbackslash}writer2latex04{\textquotedbl}}
\end{stylePx}

\begin{stylePix}
(If you have extracted to the root of drive C, you
don{\textquotesingle}t have to edit this line.)
\end{stylePix}

\begin{stylePix}
At a command line type \textstyleUserEntry{java {}-version} to verify
that the Java executable is in your path. If this is not the case or
you have several Java versions installed you should edit the next line
to contain the full path to the Java executable, eg.
\end{stylePix}

\begin{stylePx}
set
JAVAEXE={\textquotedbl}C:{\textbackslash}j2sdk1.4.0\_01{\textbackslash}bin{\textbackslash}java''
\end{stylePx}
\end{listLiiileveli}
\subsubsection{Installation for Unix and friends}
\setcounter{listLivleveli}{0}
\begin{listLivleveli}
\item 
\begin{stylePxi}
Unzip \textstyleSourceText{writer2latex04.zip} into some directory. This
will create a subdirectory \textstyleSourceText{writer2latex04}.
\end{stylePxi}
\item 
\begin{stylePxi}
Add this directory to your PATH environment variable or create a
symbolic link to the file \textstyleSourceText{w2l} from some directory
in yout PATH.
\end{stylePxi}
\item 
\begin{stylePxi}
Open the fle \textstyleSourceText{w2l} with a text editor and replace
the path at the top of the file with the full path to Writer2LaTeX, eg.
\end{stylePxi}
\end{listLivleveli}
\begin{listLiiileveli}
\begin{stylePx}
\textstyleSourceText{W2LPATH={\textquotedbl}/home/username/writer2latex04{\textquotedbl}}
\end{stylePx}

\begin{stylePix}
(If you have extracted into your home directory, you
don{\textquotesingle}t have to edit this line.)
\end{stylePix}

\begin{stylePix}
Open a command shell and type \textstyleUserEntry{java {}-version} to
verify that the Java executable is in your path. If this is not the
case or you have several Java versions installed you should edit the
next line to contain the full path to the Java executable, ie.
\end{stylePix}

\begin{stylePx}
\textstyleSourceText{set
JAVAEXE={\textquotedbl}/path/to/java/executable/''}
\end{stylePx}
\item 
\begin{stylePix}
Add execute permissions to \textstyleSourceText{w2l} as follows:
\end{stylePix}

\begin{stylePx}
chmod +x w2l
\end{stylePx}
\end{listLiiileveli}
\subsection{How to install Writer2LaTeX as an export filter}
\begin{styleTextbody}
Writer2LaTeX can work as an export filter for OOo Writer. This requires
OpenOffice.org 1.1 or StarOffice 7. It does \textit{not} work with
OpenOffice.org 1.0 or StarOffice 6.0.
\end{styleTextbody}

\begin{styleTextbody}
You can also use Writer2LaTeX as an export filter in OOo 1.9.x (should
work with recent versions).
\end{styleTextbody}

\begin{styleTextbody}
The following instructions covers all operating systems.
\end{styleTextbody}

\subsubsection{Uninstalling previous versions of Writer2LaTeX}
\begin{styleTextbody}
If you have installed a version of Writer2LaTeX \textit{prior to} 0.3.2,
you will have to undo the changes you made to the file
\textstyleSourceText{TypeDetection.xcu}:
\end{styleTextbody}

\begin{listLvleveli}
\item 
\begin{stylePxii}
If you \textit{copied} \textstyleSourceText{TypeDetection.xcu} into your
user settings you can delete or (to be safe) rename the file.
\end{stylePxii}
\item 
\begin{stylePxii}
If you \textit{edited} an existing version of
\textstyleSourceText{TypeDetection.xcu} you should restore the backup
copy of the file. If you forgot to take a backup, you will have to
delete the additions by hand (take a backup copy this time!).
\end{stylePxii}
\end{listLvleveli}
\begin{styleTextbody}
When you restart OOo, the filters should have disappeared from the
\textbf{File {--} Export} menu.
\end{styleTextbody}

\begin{styleTextbody}
See the section below for information on how to uninstall Writer2LaTeX
0.3.2 and later.
\end{styleTextbody}

\subsubsection{Installation for OOo 1.1.x}
\begin{styleTextbody}
\textit{Note:} If you have made a \textstyleTeletype{{}-net} (multiuser)
installation of OOo, you will normally need to log in as
root/administrator to install Writer2LaTeX.
\end{styleTextbody}

\begin{styleTextbody}
Before you start, you need an installation of OOo where
\end{styleTextbody}

\begin{listLvileveli}
\item 
\begin{stylePxiii}
You have set up OOo to use Java. If you didn{\textquotesingle}t do that
during installation, you can run \textstyleSourceText{{\textless}OOo
install{\textgreater}/program/jvmsetup}. (Of course this requires that
you have installed Java on your system).
\end{stylePxiii}
\item 
\begin{stylePxiii}
You must have the \textit{Mobile Device Filters installed}. If you
didn{\textquotesingle}t install these during installation
(it{\textquotesingle}s not part of a standard installation!), you can
run OOo setup, choose Modify and add the filters. This will install a
framework for Java based filters in OOo (known as xmerge), which is
also used by Writer2LaTeX (despite the fact that it has nothing to do
with mobile devices).
\end{stylePxiii}
\end{listLvileveli}
\begin{styleTextbody}
Then the installation proceeds as follows:
\end{styleTextbody}

\setcounter{listLviileveli}{0}
\begin{listLviileveli}
\item 
\begin{stylePxiv}
Copy \textstyleTeletype{writer2latex.jar},
\textstyleSourceText{xmergefix.jar} and
\textstyleTeletype{writer2latex.xml} into the classes directory 
\end{stylePxiv}

\begin{stylePxv}
{\textless}OOo install{\textgreater}/program/classes/
\end{stylePxv}
\item 
\begin{stylePxiv}
Rename the existing \textstyleSourceText{xmerge.jar} to
\textstyleSourceText{oldxmerge.jar} (or whatever you like; this is only
to have a backup of the old version).
\end{stylePxiv}
\item 
\begin{stylePxiv}
Rename \textstyleSourceText{xmergefix.jar} to
\textstyleSourceText{xmerge.jar}.
\end{stylePxiv}
\item 
\begin{stylePxiv}
Copy \textstyleSourceText{w2lfilter.zip }into the directory
\end{stylePxiv}

\begin{stylePxv}
{\textless}OOo install{\textgreater}/share/uno\_packages
\end{stylePxv}
\item 
\begin{stylePxiv}
Make sure that no OOo processes are running: Close all document windows
and (under MS Windows) the Quick Starter.
\end{stylePxiv}
\item 
\begin{stylePxiv}
From a command shell, navigate to the directory
\end{stylePxiv}

\begin{stylePxv}
{\textless}OOo install{\textgreater}/program
\end{stylePxv}

\begin{stylePxiv}
and type
\end{stylePxiv}

\begin{stylePxv}
pkgchk {}-{}-shared
\end{stylePxv}

\begin{stylePxiv}
This will register Writer2LaTeX as a filter in OOo. If it works, there
will be no messages on the screen.
\end{stylePxiv}
\item 
\begin{stylePxiv}
Now restart OOo.
\end{stylePxiv}
\end{listLviileveli}
\subsubsection{Installation for OOo 1.9.x}
\begin{styleTextbody}
\textit{Note:} If you have made a \textstyleTeletype{{}-net} (multiuser)
installation of OOo, you will normally need to log in as
root/administrator to install Writer2LaTeX.
\end{styleTextbody}

\begin{styleTextbody}
Before you start, you need an installation of OOo where
\end{styleTextbody}

\begin{listLvileveli}
\item 
\begin{stylePxiii}
You have set up OOo to use Java. If you didn{\textquotesingle}t do that
during installation, you can configure java under \textbf{Tools {--}
Options}. Of course this requires that you have installed Java on your
system.
\end{stylePxiii}
\item 
\begin{stylePxiii}
You must have the \textit{Mobile Device Filters installed}. If you
didn{\textquotesingle}t install these during installation
(it{\textquotesingle}s not part of a standard installation!), you can
run OOo setup, choose Modify and add the filters. This will install a
framework for Java based filters in OOo (known as xmerge), which is
also used by Writer2LaTeX (despite the fact that it has nothing to do
with mobile devices).
\end{stylePxiii}
\end{listLvileveli}
\begin{styleTextbody}
Then the installation proceeds as follows:
\end{styleTextbody}

\setcounter{listLviiileveli}{0}
\begin{listLviiileveli}
\item 
\begin{stylePxvi}
Copy \textstyleTeletype{writer2latex.jar} and
\textstyleTeletype{writer2latex.xml} into the classes directory 
\end{stylePxvi}

\begin{stylePxvii}
{\textless}OOo install{\textgreater}/program/classes/
\end{stylePxvii}
\item 
\begin{stylePxvi}
Make sure that no OOo processes are running: Close all document windows
and (under MS Windows) the Quick Starter.
\end{stylePxvi}
\item 
\begin{stylePxvi}
From a command shell, navigate to the directory
\end{stylePxvi}

\begin{stylePxvii}
{\textless}OOo install{\textgreater}/program
\end{stylePxvii}
\item 
\begin{stylePxvi}
and type
\end{stylePxvi}

\begin{stylePxvii}
unopkg gui
\end{stylePxvii}
\item 
\begin{stylePxvi}
Select \textbf{OpenOffice.org packages} and select
\textstyleSourceText{w2lfilters20.zip} using the \textbf{Browse}
button.
\end{stylePxvi}
\item 
\begin{stylePxvi}
Now restart OOo.
\end{stylePxvi}
\end{listLviiileveli}
\begin{styleTextbody}
If you only want to install for a single user, select \textbf{My
packages} instead {--} this is also possible from inside OOo, using the
\textbf{Tools {--} Packages} menu.
\end{styleTextbody}

\subsection{Uninstall Writer2LaTeX}
\begin{styleTextbody}
To remove the Writer2LaTeX filters from your OOo installation, you
should proceed as follows for OOo 1.1.x:
\end{styleTextbody}

\setcounter{listLviileveli}{0}
\begin{listLviileveli}
\item 
\begin{stylePxiv}
Delete \textstyleSourceText{w2lfilter.zip from} the directory
\end{stylePxiv}

\begin{stylePxv}
{\textless}OOo install{\textgreater}/share/uno\_packages
\end{stylePxv}
\item 
\begin{stylePxiv}
Make sure that no OOo processes are running: Close all document windows
and (under MS Windows) the Quick Starter.
\end{stylePxiv}
\item 
\begin{stylePxiv}
From a command shell navigate to the directory
\end{stylePxiv}

\begin{stylePxv}
{\textless}OOo install{\textgreater}/program
\end{stylePxv}

\begin{stylePxiv}
and type
\end{stylePxiv}

\begin{stylePxv}
pkgchk {}-{}-shared
\end{stylePxv}

\begin{stylePxiv}
This will remove the registration of Writer2LaTeX from OOo. If it works,
there will be no messages on the screen.
\end{stylePxiv}
\item 
\begin{stylePxiv}
Now restart OOo.
\end{stylePxiv}
\end{listLviileveli}
\begin{styleTextbody}
For OOo 1.9.x, use the \textstyleSourceText{unopkg gui} command as
described above to remove \textstyleSourceText{w2lfilters20.zip}.
\end{styleTextbody}

\begin{styleTextbody}
You may also undo the changes you have made to OOo{\textquotesingle}s
\textstyleSourceText{classes} directory, but this is not required.
\end{styleTextbody}

\section{Using Writer2LaTeX and Writer2BibTeX}
\begin{styleTextbody}
Writer2LaTeX is quite flexible: It can take advantage of several LaTeX
packages, such as \textstyleSourceText{hyperref},
\textstyleSourceText{pifont}, \textstyleUserEntry{ulem}. It can create
customized LaTeX code based on the styles used in the document. Also it
supports 25 different languages, latin, greek and cyrillic scripts and
7 inputencodings.
\end{styleTextbody}

\begin{styleTextbody}
The flexibility makes it possible to use Writer2LaTeX from several
philosophies:
\end{styleTextbody}

\begin{listLixleveli}
\item 
\begin{stylePxviii}
You can use LaTeX as a typesetting engine for your OOo documents:
Writer2LaTeX can be configured to create a LaTeX document with as much
formatting as possible preserved. Note that the resulting LaTeX source
will be readable, but not very clean.
\end{stylePxviii}

\begin{stylePxviii}
Be aware that even though Writer2LaTeX tries hard to cope with any
document, you will only get good results for well structured documents,
ie. documents that are fomatted using \textit{styles}.
\end{stylePxviii}
\item 
\begin{stylePxviii}
If you need to continue the work on your document in LaTeX your primary
interest may be the content rather than the formatting. Writer2LaTeX
can be configured to produce a LaTeX document which strips most of the
formatting and hence produces a clean LaTeX source from \textit{any}
source document.
\end{stylePxviii}
\item 
\begin{stylePxviii}
If  you don{\textquotesingle}t like to write LaTeX code by hand, you may
use OOo as a simple graphical front{}-end for LaTeX. Using a special
OOo Writer template and a special configuration file for Writer2LaTeX,
you can create well{}-structured LaTeX documents that resembles
``hand{}-written'' LaTeX documents. You can compare this to the way
\href{http://www.lyx.org/}{LyX} works.
\end{stylePxviii}

\begin{stylePxviii}
Writer2LaTeX does not provide an input filter for LaTeX. It is
recommended to use Eitan M. Gurari{\textquotesingle}s
\href{http://www.cse.ohio-state.edu/~gurari/TeX4ht/mn.html}{TeX4ht} to
convert LaTeX documents into OOo Writer format. Roundtrip editing OOo
Writer \ding{214} LaTeX is not possible in general, but
Writer2LaTeX+TeX4ht does provide some basic support for this, see
section 3.6.
\end{stylePxviii}
\end{listLixleveli}
\subsection[The LaTeX package ooomath.sty]{\label{ref:writersty}The
LaTeX package \textstyleSourceText{ooomath.sty}}
\begin{styleTextbody}
OOo Math has a few features that are not available in standard LaTeX
packages. Hence Writer2LaTeX uses an optional package
\textstyleSourceText{ooomath.sty}\footnote{This pakcage replaces
\textstyleSourceText{writer.sty} used by older versions of
Writer2LaTeX.} which implements these constructions. This packages is
only needed for documents containing formulas.
\end{styleTextbody}

\begin{styleTextbody}
It is sufficient to place \textstyleSourceText{ooomath.sty} in the same
directory as the converted LaTeX document. It will however be more
convenient if you install it in your TeX distribution. The proper place
will usually be the ``local texmf tree'', please see the documentation
of your TeX distribution. Below are specific instructions for teTeX and
MikTeX:
\end{styleTextbody}

\subsubsection{Instructions for teTeX (unix)}
\begin{styleTextbody}
If you use teTeX you can install \textstyleSourceText{ooomath.sty} as
follows:
\end{styleTextbody}

\begin{styleTextbody}
Open a shell and type
\end{styleTextbody}

\begin{stylePreformattedText}
texconfig conf
\end{stylePreformattedText}

\begin{styleTextbody}
This will list the configuration details for teTeX. Under the heading
``Kpathsea'' you will see a list of directories searched by TeX. You
can put \textstyleSourceText{ooomath.sty} in the subdirectory
\textstyleSourceText{tex} of any of these directories. Usually the
directory
\end{styleTextbody}

\begin{stylePreformattedText}
/home/{\textless}user name{\textgreater}/texmf/tex
\end{stylePreformattedText}

\begin{styleTextbody}
can be used (you can create it if it doesn{\textquotesingle}t exist).
\end{styleTextbody}

\begin{styleTextbody}
Next you should type
\end{styleTextbody}

\begin{stylePreformattedText}
texconfig rehash
\end{stylePreformattedText}

\begin{styleTextbody}
to make teTeX refresh it{\textquotesingle}s filename database.
\end{styleTextbody}

\subsubsection{Instructions for MikTeX (Windows)}
\begin{styleTextbody}
If you use MikTeX you can install \textstyleSourceText{ooomath.sty} as
follows:
\end{styleTextbody}

\begin{styleTextbody}
Copy \textstyleSourceText{ooomath.sty} to the \textstyleSourceText{tex}
subdirectory in the local texmf tree. With a standard installation this
will be the directory
\end{styleTextbody}

\begin{stylePreformattedText}
c:{\textbackslash}localtexmf{\textbackslash}tex
\end{stylePreformattedText}

\begin{styleTextbody}
If this directory does not exist you should start ``MikTeX Options''
(you can find this in the Start Menu). On the tab page \textbf{Roots}
you can see the location of the local texmf tree.
\end{styleTextbody}

\begin{styleTextbody}
Next you should start ``MikTeX Options''. On the tab page
\textbf{General}, click the button \textbf{Refresh Now} to make MikTeX
refresh it{\textquotesingle}s filename database.
\end{styleTextbody}

\subsection{Converting to LaTeX from the command line}
\begin{styleTextbody}
To convert a file to LaTeX use the command line
\end{styleTextbody}

\begin{stylePreformattedText}
w2l [{}-latex] [{}-config {\textless}configfile{\textgreater}]
{\textless}writer document to convert{\textgreater} [{\textless}output
path and/or file name{\textgreater}]
\end{stylePreformattedText}

\begin{styleTextbody}
The parts in square brackets are optional.
\end{styleTextbody}

\begin{styleTextbody}
This will produce a LaTeX file with the specified name. If no output
file is specified, Writer2LaTeX will use the same name as the original
document, but change the extension to \textstyleTeletype{.tex}.
\end{styleTextbody}

\begin{styleTextbody}
Examples:
\end{styleTextbody}

\begin{stylePreformattedText}
w2l mydocument.sxw mypath/myoutputdocument.tex
\end{stylePreformattedText}

\begin{styleTextbody}
or
\end{styleTextbody}

\begin{stylePreformattedText}
w2l {}-config clean.xml mydocument.sxw
\end{stylePreformattedText}

\begin{styleTextbody}
If you specify the \textstyleTeletype{{}-config} option, Writer2LaTeX
will load this configuration file before converting your document. You
can read more about configuration in section 3.5.
\end{styleTextbody}

\begin{styleTextbody}
The script \textstyleSourceText{w2l} also provides a shorthand notation
to use the sample configuration files included in
\textstyleSourceText{writer2latex04.zip}. The command line is
\end{styleTextbody}

\begin{stylePreformattedText}
w2l
[{}-ultraclean{\textbar}{}-clean{\textbar}{}-pdfscreen{\textbar}{}-pdfprint{\textbar}{}-article]
{\textless}writer document to convert{\textgreater} [{\textless}output
path and/or file name{\textgreater}]
\end{stylePreformattedText}

\begin{styleTextbody}
For example to produce a clean LaTeX file (ie. ignoring most of the
formatting from the source document):
\end{styleTextbody}

\begin{stylePreformattedText}
w2l {}-clean mydocument.sxw
\end{stylePreformattedText}

\begin{styleTextbody}
It is recommended that you extend \textstyleSourceText{w2l} /
\textstyleSourceText{w2l.bat} to support your own configuration files.
\end{styleTextbody}

\subsection{Converting to BibTeX from the command line}
\begin{styleTextbody}
Writer2BibTeX extracts bibliography data to a BibTeX file. To do this
use the commandline
\end{styleTextbody}

\begin{stylePreformattedText}
w2l {}-bibtex {\textless}writer document to convert{\textgreater}
[{\textless}output path and/or file name{\textgreater}]
\end{stylePreformattedText}

\begin{styleTextbody}
You can also extract the data as part of the conversion to LaTeX, see
section 3.5.
\end{styleTextbody}

\subsection{Using Writer2LaTeX and Writer2BibTeX as export filters}
\begin{styleTextbody}
If you choose \textbf{File {--} Export} in Writer you should be able to
choose \textbf{LaTeX 2e}, \textbf{BibTeX data file} as file type. 
\end{styleTextbody}

\begin{styleTextbody}
\textbf{Note:} You have to use the export menu because there is no
import filter for LaTeX/BibTeX. You should always save in the native
format of OOo as well!
\end{styleTextbody}

\begin{styleTextbody}
\textbf{Note}: Currently embedded graphics are not converted when
Writer2LaTeX is used as an export filter. Also using Writer2BibTeX in
conjunction with Writer2LaTeX is currently only possible from the
command line. This is because of an issue with xmerge. A fix for this
is planned for a later version of Writer2LaTeX.
\end{styleTextbody}

\subsection{\label{ref:latexconfig}Configuration}
\begin{styleTextbody}
LaTeX export can be configured with a configuration file. The
configuration is read from several sources:
\end{styleTextbody}

\begin{listLxleveli}
\item 
\begin{stylePxix}
First Writer2LaTeX reads the file \textstyleTeletype{writer2latex.xml}
in the same directory as \textstyleTeletype{writer2latex.jar}. This
file is supposed to contain installation{}-wide configuration.
\end{stylePxix}
\item 
\begin{stylePxix}
Then it reads the file \textstyleSourceText{writer2latex.xml} in your
home directory (unix, eg. \textstyleSourceText{/home/username}) or user
profile (windows, eg. \textstyleSourceText{c:{\textbackslash}documents
and settings{\textbackslash}username}). This file is supposed to
contain user{}-specific configuration. The installation{}-wide
configuration may specify, that this file should be generated
automatically.
\end{stylePxix}
\item 
\begin{stylePxix}
Finally, the documentation file you specify on the command line will be
read.
\end{stylePxix}
\end{listLxleveli}
\begin{styleTextbody}
The configuration file is an xml file; these are the default contents:
\end{styleTextbody}

\begin{stylePreformattedText}
{\textless}?xml version={\textquotedbl}1.0{\textquotedbl}
encoding={\textquotedbl}UTF{}-8{\textquotedbl} ?{\textgreater}
\end{stylePreformattedText}

\begin{stylePreformattedText}
{\textless}config{\textgreater}
\end{stylePreformattedText}

\begin{stylePreformattedText}
 {\textless}option
name={\textquotedbl}create\_user\_config{\textquotedbl}
value={\textquotedbl}true{\textquotedbl} /{\textgreater}
\end{stylePreformattedText}

\begin{stylePreformattedText}
 {\textless}option name={\textquotedbl}backend{\textquotedbl}
value={\textquotedbl}generic{\textquotedbl} /{\textgreater}
\end{stylePreformattedText}

\begin{stylePreformattedText}
 {\textless}option name={\textquotedbl}no\_preamble{\textquotedbl}
value={\textquotedbl}false{\textquotedbl} /{\textgreater}
\end{stylePreformattedText}

\begin{stylePreformattedText}
 {\textless}option name={\textquotedbl}documentclass{\textquotedbl}
value={\textquotedbl}article{\textquotedbl} /{\textgreater}
\end{stylePreformattedText}

\begin{stylePreformattedText}
 {\textless}option name={\textquotedbl}global\_options{\textquotedbl}
value={\textquotedbl}{\textquotedbl} /{\textgreater}
\end{stylePreformattedText}

\begin{stylePreformattedText}
 {\textless}option name={\textquotedbl}inputencoding{\textquotedbl}
value={\textquotedbl}ascii{\textquotedbl} /{\textgreater}
\end{stylePreformattedText}

\begin{stylePreformattedText}
 {\textless}option name={\textquotedbl}multilingual{\textquotedbl}
value={\textquotedbl}true{\textquotedbl} /{\textgreater}
\end{stylePreformattedText}

\begin{stylePreformattedText}
 {\textless}option name={\textquotedbl}greek\_math{\textquotedbl}
value={\textquotedbl}true{\textquotedbl} /{\textgreater}
\end{stylePreformattedText}

\begin{stylePreformattedText}
 {\textless}option name={\textquotedbl}use\_ooomath{\textquotedbl}
value={\textquotedbl}false{\textquotedbl} /{\textgreater}
\end{stylePreformattedText}

\begin{stylePreformattedText}
 {\textless}option name={\textquotedbl}use\_pifont{\textquotedbl}
value={\textquotedbl}false{\textquotedbl} /{\textgreater}
\end{stylePreformattedText}

\begin{stylePreformattedText}
 {\textless}option name={\textquotedbl}use\_ifsym{\textquotedbl}
value={\textquotedbl}false{\textquotedbl} /{\textgreater}
\end{stylePreformattedText}

\begin{stylePreformattedText}
 {\textless}option name={\textquotedbl}use\_wasysym{\textquotedbl}
value={\textquotedbl}false{\textquotedbl} /{\textgreater}
\end{stylePreformattedText}

\begin{stylePreformattedText}
 {\textless}option name={\textquotedbl}use\_bbding{\textquotedbl}
value={\textquotedbl}false{\textquotedbl} /{\textgreater}
\end{stylePreformattedText}

\begin{stylePreformattedText}
 {\textless}option name={\textquotedbl}use\_eurosym{\textquotedbl}
value={\textquotedbl}false{\textquotedbl} /{\textgreater}
\end{stylePreformattedText}

\begin{stylePreformattedText}
 {\textless}option name={\textquotedbl}use\_tipa{\textquotedbl}
value={\textquotedbl}false{\textquotedbl} /{\textgreater}
\end{stylePreformattedText}

\begin{stylePreformattedText}
 {\textless}option name={\textquotedbl}use\_color{\textquotedbl}
value={\textquotedbl}true{\textquotedbl} /{\textgreater}
\end{stylePreformattedText}

\begin{stylePreformattedText}
 {\textless}option name={\textquotedbl}use\_hyperref{\textquotedbl}
value={\textquotedbl}true{\textquotedbl} /{\textgreater}
\end{stylePreformattedText}

\begin{stylePreformattedText}
 {\textless}option name={\textquotedbl}use\_endnotes{\textquotedbl}
value={\textquotedbl}false{\textquotedbl} /{\textgreater}
\end{stylePreformattedText}

\begin{stylePreformattedText}
 {\textless}option name={\textquotedbl}use\_ulem{\textquotedbl}
value={\textquotedbl}false{\textquotedbl} /{\textgreater}
\end{stylePreformattedText}

\begin{stylePreformattedText}
 {\textless}option name={\textquotedbl}use\_lastpage{\textquotedbl}
value={\textquotedbl}false{\textquotedbl} /{\textgreater}
\end{stylePreformattedText}

\begin{stylePreformattedText}
 {\textless}option name={\textquotedbl}use\_bibtex{\textquotedbl}
value={\textquotedbl}false{\textquotedbl} /{\textgreater}
\end{stylePreformattedText}

\begin{stylePreformattedText}
 {\textless}option name={\textquotedbl}bibtex\_style{\textquotedbl}
value={\textquotedbl}plain{\textquotedbl} /{\textgreater}
\end{stylePreformattedText}

\begin{stylePreformattedText}
 {\textless}option name={\textquotedbl}formatting{\textquotedbl}
value={\textquotedbl}convert\_basic{\textquotedbl} /{\textgreater}
\end{stylePreformattedText}

\begin{stylePreformattedText}
 {\textless}option name={\textquotedbl}page\_formatting{\textquotedbl}
value={\textquotedbl}convert\_all{\textquotedbl} /{\textgreater}
\end{stylePreformattedText}

\begin{stylePreformattedText}
 {\textless}option
name={\textquotedbl}ignore\_hard\_page\_breaks{\textquotedbl}
value={\textquotedbl}false{\textquotedbl} /{\textgreater}
\end{stylePreformattedText}

\begin{stylePreformattedText}
 {\textless}option
name={\textquotedbl}ignore\_hard\_line\_breaks{\textquotedbl}
value={\textquotedbl}false{\textquotedbl} /{\textgreater}
\end{stylePreformattedText}

\begin{stylePreformattedText}
 {\textless}option
name={\textquotedbl}ignore\_empty\_paragraphs{\textquotedbl}
value={\textquotedbl}false{\textquotedbl} /{\textgreater}
\end{stylePreformattedText}

\begin{stylePreformattedText}
 {\textless}option
name={\textquotedbl}ignore\_double\_spaces{\textquotedbl}
value={\textquotedbl}false{\textquotedbl} /{\textgreater}
\end{stylePreformattedText}

\begin{stylePreformattedText}
 {\textless}option name={\textquotedbl}debug{\textquotedbl}
value={\textquotedbl}false{\textquotedbl} /{\textgreater}
\end{stylePreformattedText}

\begin{stylePreformattedText}
 {\textless}heading{}-map
max{}-level={\textquotedbl}5{\textquotedbl}{\textgreater}
\end{stylePreformattedText}

\begin{stylePreformattedText}
 {\textless}heading{}-level{}-map
writer{}-level={\textquotedbl}1{\textquotedbl}
name={\textquotedbl}section{\textquotedbl}
level={\textquotedbl}1{\textquotedbl} /{\textgreater}
\end{stylePreformattedText}

\begin{stylePreformattedText}
 {\textless}heading{}-level{}-map
writer{}-level={\textquotedbl}2{\textquotedbl}
name={\textquotedbl}subsection{\textquotedbl}
level={\textquotedbl}2{\textquotedbl} /{\textgreater}
\end{stylePreformattedText}

\begin{stylePreformattedText}
 {\textless}heading{}-level{}-map
writer{}-level={\textquotedbl}3{\textquotedbl}
name={\textquotedbl}subsubsection{\textquotedbl}
level={\textquotedbl}3{\textquotedbl} /{\textgreater}
\end{stylePreformattedText}

\begin{stylePreformattedText}
 {\textless}heading{}-level{}-map
writer{}-level={\textquotedbl}4{\textquotedbl}
name={\textquotedbl}paragraph{\textquotedbl}
level={\textquotedbl}4{\textquotedbl} /{\textgreater}
\end{stylePreformattedText}

\begin{stylePreformattedText}
 {\textless}heading{}-level{}-map
writer{}-level={\textquotedbl}5{\textquotedbl}
name={\textquotedbl}subparagraph{\textquotedbl}
level={\textquotedbl}5{\textquotedbl} /{\textgreater}
\end{stylePreformattedText}

\begin{stylePreformattedText}
 {\textless}/heading{}-map{\textgreater}
\end{stylePreformattedText}

\begin{stylePreformattedText}
 {\textless}custom{}-preamble /{\textgreater}
\end{stylePreformattedText}

\begin{stylePreformattedText}
{\textless}/config{\textgreater}
\end{stylePreformattedText}

\begin{styleTextbody}
The meaning of each part is explained in the following sections.
Writer2LaTeX comes with five sample configuration files:
\end{styleTextbody}

\begin{listLxileveli}
\item 
\begin{stylePxx}
\textstyleSourceText{ultraclean.xml} to produce a \textit{clean} LaTeX
file, ie. almost all the formatting is ignored.
\end{stylePxx}
\item 
\begin{stylePxx}
\textstyleSourceText{clean.xml} is a less radical version; preserves
hyperlinks, color and most character formatting.
\end{stylePxx}
\item 
\begin{stylePxx}
\textstyleSourceText{pdfscreen.xml} to produce a LaTeX file which is
optimized for screen viewing using the package
\textstyleSourceText{pdfscreen.sty}.
\end{stylePxx}
\item 
\begin{stylePxx}
\textstyleSourceText{pdfprint.xml} to produce a LaTeX file which is
optimized for printing with pdfTeX.
\end{stylePxx}
\item 
\begin{stylePxx}
\textstyleSourceText{article.xml} to produce a LaTeX article, see
section 3.6.
\end{stylePxx}
\end{listLxileveli}
\subsubsection{Basic options}
\begin{listLxiileveli}
\item 
\begin{stylePxxi}
If the option\textstyleUserEntry{ create\_user\_config }if set to
\textstyleUserEntry{true}, the user specific configuration file
mentioned above will be created if it does not exist.
\end{stylePxxi}
\item 
\begin{stylePxxi}
The option \textstyleSourceText{backend} can have any of the values
\textstyleTeletype{generic} (default), \textstyleTeletype{dvips} or
\textstyleTeletype{pdftex}. This will create LaTeX files suitable for
any backend/dvi driver, dvips or pdfTeX respectively.
\end{stylePxxi}
\item 
\begin{stylePxxi}
If the option \textstyleTeletype{no\_preamble} is set to
\textstyleTeletype{false}, Writer2LaTeX will not create the a LaTeX
preamble, nor include
\textstyleTeletype{{\textbackslash}begin\{document\}} and
\textstyleTeletype{{\textbackslash}end\{document\}}. This is useful if
the document is to be included in another LaTeX document. Note that in
this case you will have to make sure that all packages/definitions
needed are available in the master LaTeX document.
\end{stylePxxi}
\end{listLxiileveli}
\begin{listLxiiileveli}
\item 
\begin{stylePxxii}
The option \textstyleTeletype{inputencoding} can have any of the values
\textstyleTeletype{ascii} (default), \textstyleTeletype{latin1},
\textstyleTeletype{latin2}, \textstyleTeletype{iso{}-8859{}-7},
\textstyleTeletype{cp1250}, \textstyleTeletype{cp1251},
\textstyleTeletype{koi8{}-r }or \textstyleTeletype{utf8}. The latter
requires Dominique Unruh{\textquotesingle}s
\textstyleTeletype{ucs.sty}. 
\end{stylePxxii}
\item 
\begin{stylePxxii}
If the option \textstyleSourceText{multilingual} is set to false,
Writer2LaTeX will assume that the document is written in one language
only {--} otherwise all the language information contained in the
document will be used.
\end{stylePxxii}
\end{listLxiiileveli}
\subsubsection{Options for document structure}
\begin{listLxiileveli}
\item 
\begin{stylePxxi}
The option \textstyleSourceText{documentclass} is the name of the
documentclass to use (default is \textstyleSourceText{article}).
\end{stylePxxi}
\item 
\begin{stylePxxi}
The option \textstyleSourceText{global\_options} is a list of global
options to add to the documentclass (the default value is an empty
string).
\end{stylePxxi}
\item 
\begin{stylePxxi}
The \textstyleSourceText{heading\_map} section specifies how headings in
OOo should map to LaTeX. Eg. the first line specifies that
\textbf{Heading 1} should map to
\textstyleSourceText{{\textbackslash}section}, which is of level 1 in
LaTeX. Up to 10~levels are supported (the same number as in OOo).
\end{stylePxxi}
\end{listLxiileveli}
\subsubsection{Font and symbol options}
\begin{listLxivleveli}
\item 
\begin{stylePxxiii}
The option \textstyleTeletype{greek\_math} can have the values
\textstyleTeletype{true} (default) or \textstyleTeletype{false}. This
means that greek letters in latin or cyrillic text are rendered in math
mode. This behaviour assumes that greek letters are used as symbols in
this context, and has the advantage that greek text fonts are not
required. It is \textit{not} used in greek text, where it would be
awful.
\end{stylePxxiii}
\item 
\begin{stylePxxiii}
The option \textstyleTeletype{use\_ooomath} can  have the values
\textstyleTeletype{true} or \textstyleTeletype{false} (default). This
enables the use of the LaTeX package \textstyleTeletype{ooomath.sty}.
If this package is not used, the necessary definitions will be included
in the LaTeX preamble, which may become quite long {--} so using
\textstyleTeletype{ooomath.sty} is recommended.
\end{stylePxxiii}
\item 
\begin{stylePxxiii}
The option \textstyleTeletype{use\_pifont} can have the values
\textstyleTeletype{true} or \textstyleTeletype{false} (default). This
enables the use of \textit{Zapf Dingbats} using the LaTeX package
\textstyleTeletype{pifont.sty}.
\end{stylePxxiii}
\item 
\begin{stylePxxiii}
The option \textstyleTeletype{use\_wasysym} can have the values
\textstyleTeletype{true} or \textstyleTeletype{false} (default). This
enables the use of the \textit{wasy} symbol font using the LaTeX
package \textstyleTeletype{wasysym.sty}.
\end{stylePxxiii}
\item 
\begin{stylePxxiii}
The option \textstyleTeletype{use\_ifsym} can have the values
\textstyleTeletype{true} or \textstyleTeletype{false} (default). This
enables the use of  the \textit{ifsym} symbol font using the LaTeX
package \textstyleTeletype{ifsym.sty}.
\end{stylePxxiii}
\item 
\begin{stylePxxiii}
The option \textstyleTeletype{use\_bbding} can have the values
\textstyleTeletype{true} or \textstyleTeletype{false} (default). This
enables the use of the \textit{bbding} symbol font (a clone of Zapf
Dingbats) using the LaTeX package \textstyleTeletype{bbding.sty}.
\end{stylePxxiii}
\item 
\begin{stylePxxiii}
The option \textstyleTeletype{use\_eurosym} can have the values
\textstyleTeletype{true} or \textstyleTeletype{false} (default). This
enables the use of the \textit{eurosym} symbol font using the LaTeX
package \textstyleTeletype{eurosym.sty}.
\end{stylePxxiii}
\item 
\begin{stylePxxiii}
The option \textstyleTeletype{use\_tipa} can have the values
\textstyleTeletype{true} or \textstyleTeletype{false} (default). This
enables the use of phonetic symbols using the LaTeX package
\textstyleTeletype{tipa.sty}.
\end{stylePxxiii}
\end{listLxivleveli}
\subsubsection{Options for other packages}
\begin{listLxvleveli}
\item 
\begin{stylePxxiv}
The option \textstyleSourceText{use\_hyperref} can have the values
\textstyleSourceText{true} (default) or \textstyleSourceText{false}.
This enables use of the package \textstyleSourceText{hyperref.sty} to
include hyperlinks in the LaTeX document.
\end{stylePxxiv}
\item 
\begin{stylePxxiv}
The option \textstyleSourceText{use\_color} can have the values
\textstyleSourceText{true} (default) or \textstyleSourceText{false}.
This enables use of the package \textstyleSourceText{hyperref.sty} to
apply color in the LaTeX document.
\end{stylePxxiv}
\item 
\begin{stylePxxiv}
The option \textstyleSourceText{use\_endnotes} can have the values
\textstyleSourceText{true} or \textstyleSourceText{false} (default).
This enables use of the package \textstyleSourceText{endnotes.sty} to
include endnotes in the LaTeX document. If set to
\textstyleSourceText{false}, endnotes will be converted to footnotes.
\end{stylePxxiv}
\item 
\begin{stylePxxiv}
The option \textstyleSourceText{use\_ulem} can have the values
\textstyleSourceText{true} or \textstyleSourceText{false} (default).
This enables use of the package \textstyleSourceText{ulem.sty} to
support underlining and crossing out in the LaTeX document.
\end{stylePxxiv}
\item 
\begin{stylePxxiv}
The option \textstyleSourceText{use\_lastpage} can have the values
\textstyleSourceText{true} or \textstyleSourceText{false} (default).
This enables use of the package \textstyleSourceText{lastpage.sty} to
represent the page count.
\end{stylePxxiv}
\end{listLxvleveli}
\subsubsection{Options for BibTeX}
\begin{listLxvileveli}
\item 
\begin{stylePxxv}
The option \textstyleTeletype{use\_bibtex} can have the values 
\textstyleTeletype{true} or \textstyleTeletype{false} (default). This
enables the use of BibTeX for bibliography generation. If it is set to
\textstyleSourceText{false}, the bibliography is included as text.
\end{stylePxxv}
\item 
\begin{stylePxxv}
The option \textstyleTeletype{bibtex\_style} can have any BibTeX style
as value (default is \textstyleTeletype{plain}). This is the BibTeX
style to be used in the LaTeX document.
\end{stylePxxv}
\end{listLxvileveli}
\subsubsection{Options to control export of formatting}
\begin{styleTextbody}
In Writer, formatting is controlled by styles. You can control how much
formatting is exported using the following options\footnote{Note that
these options have changed a lot since version 0.3.2.}. Note that these
options has a major impact on the structure of the LaTeX document
created.
\end{styleTextbody}

\begin{listLxviileveli}
\item 
\begin{stylePxxvi}
The option \textstyleTeletype{formatting}\footnote{This option replaces
the options \textstyleTeletype{character\_formatting},
\textstyleTeletype{paragraph\_formatting},
\textstyleTeletype{list\_formatting},
\textstyleTeletype{heading\_formatting} and
\textstyleTeletype{ignore\_footnotes\_configuration} used in previous
versions of Writer2LaTeX.} can have any of these values:
\end{stylePxxvi}

\begin{listLxviilevelii}
\item 
\begin{stylePxxvi}
\textstyleTeletype{ignore\_all} will instruct Writer2LaTeX to ignore
\textit{all} character, paragraph, heading, list and footnote
formatting contained in the document.
\end{stylePxxvi}
\item 
\begin{stylePxxvi}
\textstyleTeletype{ignore\_most} will preserve basic character
formatting.
\end{stylePxxvi}
\item 
\begin{stylePxxvi}
\textstyleTeletype{convert\_basic} (default) will preserve basic
character formatting as well as all numberings (lists, headings,
footnotes).
\end{stylePxxvi}
\item 
\begin{stylePxxvi}
\textstyleTeletype{convert\_most} will convert all supported formatting,
except that paragraph formatting and font size is only converted if it
is set by a style. To be able to preserve formatting, an environment is
created for all paragraph styles, custom lists is used for listings,
headings is reformatted using the
\textstyleSourceText{{\textbackslash}@startsection} command etc.
\end{stylePxxvi}
\item 
\begin{stylePxxvi}
\textstyleTeletype{convert\_all} will preserve \textit{all} supported
formatting.
\end{stylePxxvi}
\end{listLxviilevelii}
\item 
\begin{stylePxxvi}
The option \textstyleSourceText{page\_formatting} can have any of the
values \textstyleSourceText{ignore\_all},
\textstyleSourceText{convert\_header\_footer},
\textstyleSourceText{convert\_all}. This will ignore all page
formatting, convert the header and footer (using custom page styles) or
convert all supported formatting (using more elaborate custom page
styles).
\end{stylePxxvi}
\item 
\begin{stylePxxvi}
The option \textstyleSourceText{ignore\_empty\_paragraphs} can have the
values \textstyleSourceText{true} (default) or
\textstyleSourceText{false}. Setting the option to
\textstyleSourceText{true} will instruct Writer2LaTeX to ignore empty
paragraphs; otherwise they are converted to
\textstyleSourceText{{\textbackslash}bigskip}.
\end{stylePxxvi}
\item 
\begin{stylePxxvi}
The option \textstyleSourceText{ignore\_double\_spaces} can have the
values \textstyleSourceText{true} (default) or
\textstyleSourceText{false}. Setting the option to
\textstyleSourceText{true} will instruct Writer2LaTeX to ignore double
spaces, otherwise they are converted to
(\textstyleSourceText{{\textbackslash} }).
\end{stylePxxvi}
\item 
\begin{stylePxxvi}
The option \textstyleSourceText{ignore\_hard\_page\_breaks} can have the
values \textstyleSourceText{true} or \textstyleSourceText{false}
(default).  Setting the option to \textstyleSourceText{true} will
instruct Writer2LaTeX to ignore hard page breaks (but not soft page
breaks specified in paragraph styles).
\end{stylePxxvi}
\item 
\begin{stylePxxvi}
The option \textstyleSourceText{ignore\_hard\_line\_breaks} can have the
values \textstyleSourceText{true} or \textstyleSourceText{false}
(default).  Setting the option to \textstyleSourceText{true} will
instruct Writer2LaTeX to ignore hard line breaks (shift{}-Enter).
\end{stylePxxvi}
\end{listLxviileveli}
\subsubsection{Style maps}
\begin{styleTextbody}
In addition you can specify maps from styles in Writer to your own LaTeX
styles in the configuration. Currently this is possible for text
styles, paragraph styles and list styles. The following examples are
from the sample configuration file \textstyleSourceText{article.xml}.
\end{styleTextbody}

\begin{styleTextbody}
This is a simple rule, that maps text formatted with the text style
\textbf{Emphasis} to the LaTeX code
\textstyleSourceText{{\textbackslash}emph\{...\}}:
\end{styleTextbody}

\begin{stylePreformattedText}
 {\textless}style{}-map name={\textquotedbl}Emphasis{\textquotedbl}
class={\textquotedbl}text{\textquotedbl}
before={\textquotedbl}{\textbackslash}emph\{{\textquotedbl}
after={\textquotedbl}\}{\textquotedbl} /{\textgreater}
\end{stylePreformattedText}

\begin{styleTextbody}
This is another simple rule, that maps paragraphs formatted with the
paragraph style \textbf{part} to the LaTeX code
\textstyleSourceText{{\textbackslash}part\{...\}}. The attribute
\textstyleSourceText{line{}-break} ensures that no line breaks are
inserted between the code and the text.
\end{styleTextbody}

\begin{stylePreformattedText}
{\textless}style{}-map name={\textquotedbl}part{\textquotedbl}
class={\textquotedbl}paragraph{\textquotedbl}
before={\textquotedbl}{\textbackslash}part\{{\textquotedbl}
after={\textquotedbl}\}{\textquotedbl}
line{}-break={\textquotedbl}false{\textquotedbl} /{\textgreater}
\end{stylePreformattedText}

\begin{styleTextbody}
This is a rule, that maps paragraph formatted with style
\textbf{Preformatted Text} to the LaTeX environment
\textstyleSourceText{verbatim}. The attribute
\textstyleSourceText{verbatim} ensures that the content of the
paragraph is exported verbatim (this implies that characters not
available in the \textstyleSourceText{inputenc} are converted to
question marks and that other content is discarded, eg. footnotes). The
\textstyleSourceText{paragraph{}-block} entry specifies code to go
before and after an entire block of paragraphs. The
\textstyleSourceText{name} attribute specifies the style of the first
paragraph; the \textstyleSourceText{next} attribute specifies the
style(s) of subsequent paragraphs in the block.
\end{styleTextbody}

\begin{stylePreformattedText}
 {\textless}style{}-map name={\textquotedbl}Preformatted
Text{\textquotedbl}
class={\textquotedbl}paragraph{}-block{\textquotedbl}
next={\textquotedbl}Preformatted Text{\textquotedbl}
before={\textquotedbl}{\textbackslash}begin\{verbatim\}{\textquotedbl}
after={\textquotedbl}{\textbackslash}end\{verbatim\}{\textquotedbl}
/{\textgreater}
\end{stylePreformattedText}

\begin{stylePreformattedText}
{\textless}style{}-map name={\textquotedbl}Preformatted
Text{\textquotedbl} class={\textquotedbl}paragraph{\textquotedbl}
before={\textquotedbl}{\textquotedbl}
after={\textquotedbl}{\textquotedbl}
verbatim={\textquotedbl}true{\textquotedbl} /{\textgreater}
\end{stylePreformattedText}

\begin{styleTextbody}
This is a more elaborate set of rules, that maps paragraphs formatted
with styles \textbf{Title}, \textbf{author} and \textbf{date} (in any
order) to \textstyleSourceText{{\textbackslash}maketitle} in LaTeX.
\end{styleTextbody}

\begin{stylePreformattedText}
{\textless}style{}-map name={\textquotedbl}Title{\textquotedbl}
class={\textquotedbl}paragraph{\textquotedbl}
before={\textquotedbl}{\textbackslash}title\{{\textquotedbl}
after={\textquotedbl}\}{\textquotedbl}
line{}-break={\textquotedbl}false{\textquotedbl}
/{\textgreater}\newline
  {\textless}style{}-map name={\textquotedbl}author{\textquotedbl}
class={\textquotedbl}paragraph{\textquotedbl}
before={\textquotedbl}{\textbackslash}author\{{\textquotedbl}
after={\textquotedbl}\}{\textquotedbl}
line{}-break={\textquotedbl}false{\textquotedbl}
/{\textgreater}\newline
  {\textless}style{}-map name={\textquotedbl}date{\textquotedbl}
class={\textquotedbl}paragraph{\textquotedbl}
before={\textquotedbl}{\textbackslash}date\{{\textquotedbl}
after={\textquotedbl}\}{\textquotedbl}
line{}-break={\textquotedbl}false{\textquotedbl}
/{\textgreater}\newline
  {\textless}style{}-map name={\textquotedbl}Title{\textquotedbl}
class={\textquotedbl}paragraph{}-block{\textquotedbl}
next={\textquotedbl}author;date{\textquotedbl}
before={\textquotedbl}{\textquotedbl}
after={\textquotedbl}{\textbackslash}maketitle{\textquotedbl}
/{\textgreater}\newline
  {\textless}style{}-map name={\textquotedbl}author{\textquotedbl}
class={\textquotedbl}paragraph{}-block{\textquotedbl}
next={\textquotedbl}Title;date{\textquotedbl}
before={\textquotedbl}{\textquotedbl}
after={\textquotedbl}{\textbackslash}maketitle{\textquotedbl}
/{\textgreater}\newline
  {\textless}style{}-map name={\textquotedbl}date{\textquotedbl}
class={\textquotedbl}paragraph{}-block{\textquotedbl}
next={\textquotedbl}Title;author{\textquotedbl}
before={\textquotedbl}{\textquotedbl}
after={\textquotedbl}{\textbackslash}maketitle{\textquotedbl}
/{\textgreater}
\end{stylePreformattedText}

\begin{styleTextbody}
This will produce code like this:
\end{styleTextbody}

\begin{stylePreformattedText}
{\textbackslash}title\{Configuration\}
\end{stylePreformattedText}

\begin{stylePreformattedText}
{\textbackslash}author\{Henrik Just\}
\end{stylePreformattedText}

\begin{stylePreformattedText}
{\textbackslash}date\{2004\}
\end{stylePreformattedText}

\begin{stylePreformattedText}
{\textbackslash}maketitle
\end{stylePreformattedText}

\begin{styleTextbody}
The last example maps a paragraph formatted with the \textbf{theorem}
list style to a LaTeX environment named \textstyleSourceText{theorem}.
Note that there are two entries for a list style: The first one to
specify the LaTeX code to put before and after the entire list. The
second one to specify the LaTeX code to put before and after each list
item.
\end{styleTextbody}

\begin{stylePreformattedText}
{\textless}style{}-map name={\textquotedbl}theorem{\textquotedbl}
class={\textquotedbl}paragraph{\textquotedbl}
before={\textquotedbl}{\textquotedbl}
after={\textquotedbl}{\textquotedbl} /{\textgreater}\newline
 {\textless}style{}-map name={\textquotedbl}theorem{\textquotedbl}
class={\textquotedbl}list{\textquotedbl}
before={\textquotedbl}{\textquotedbl}
after={\textquotedbl}{\textquotedbl} /{\textgreater}\newline
 {\textless}style{}-map name={\textquotedbl}theorem{\textquotedbl}
class={\textquotedbl}listitem{\textquotedbl}
before={\textquotedbl}{\textbackslash}begin\{theorem\}{\textquotedbl}
after={\textquotedbl}{\textbackslash}end\{theorem\}{\textquotedbl}
/{\textgreater}
\end{stylePreformattedText}

\begin{styleTextbody}
When you override a style, all formatting specified in the original
document will be igored.
\end{styleTextbody}

\subsubsection{Math symbols}
\begin{styleTextbody}
In OOo Math you can add user{}-defined symbols. Writer2LaTeX already
understands the predefined symbols such as
\textstyleSourceText{\%alpha}. If you define your own symbols, you can
add an entry in the confguration that specifies LaTeX code to use. The
\textstyleSourceText{math{}-symbol{}-map} element is used for this:
\end{styleTextbody}

\begin{stylePreformattedText}
{\textless}math{}-symbol{}-map name=''ddarrow''
latex=''{\textbackslash}Downarrow'' /{\textgreater}
\end{stylePreformattedText}

\begin{styleTextbody}
This example will map the symbol\textstyleSourceText{ \%ddarrow} to the
LaTeX code \textstyleSourceText{{\textbackslash}Downarrow}.
\end{styleTextbody}

\subsubsection{Custom preamble}
\begin{styleTextbody}
The text you specify in the element
\textstyleSourceText{custom{}-preamble} will be copied verbatim into
the LaTeX preamble. For example:
\end{styleTextbody}

\begin{stylePreformattedText}
{\textless}custom{}-preamble{\textgreater}{\textbackslash}usepackage\{palatino\}{\textless}/custom{}-preamble{\textgreater}
\end{stylePreformattedText}

\begin{styleTextbody}
to typeset your document using the postscript font palatino.
\end{styleTextbody}

\subsection{\label{ref:latexfrontend}Using OpenOffice.org as a frontend
for LaTeX}
\begin{styleTextbody}
Writer2LaTeX has some simple support for using OOo as a frontend for
LaTeX. The long term goal of this is to turn Writer into a
near{}-wysiwyg LaTeX editor somewhat like LyX.
\end{styleTextbody}

\begin{styleTextbody}
Here is a short description:
\end{styleTextbody}

\begin{styleTextbody}
Create a new document based on the template LaTeX{}-article.stw.
\end{styleTextbody}

\begin{styleTextbody}
This template contains a number of styles that corresponds to LaTeX
code:
\end{styleTextbody}

\begin{longtable}[c]{|p{5.467cm}|p{5.467cm}|p{5.467cm}|}
\hline
\begin{minipage}[c]{5.467cm}\begin{styleTableHeading}
OOo Writer style
\end{styleTableHeading}
\end{minipage}&
\begin{minipage}[c]{5.467cm}\begin{styleTableHeading}
OOo Writer class
\end{styleTableHeading}
\end{minipage}&
\begin{minipage}[c]{5.467cm}\begin{styleTableHeading}
LaTeX code
\end{styleTableHeading}
\end{minipage}\\\hline
\endhead
\begin{minipage}[c]{5.467cm}\begin{styleTableContents}
\textit{Title}\footnotemark{}
\end{styleTableContents}
\end{minipage}&
\begin{minipage}[c]{5.467cm}\begin{styleTableContents}
paragraph
\end{styleTableContents}
\end{minipage}&
\begin{minipage}[c]{5.467cm}\begin{styleTableContents}
\textstyleSourceText{{\textbackslash}title\{...\}}\footnotemark{}
\end{styleTableContents}
\end{minipage}\\\hline
\begin{minipage}[c]{5.467cm}\begin{styleTableContents}
author
\end{styleTableContents}
\end{minipage}&
\begin{minipage}[c]{5.467cm}\begin{styleTableContents}
paragraph
\end{styleTableContents}
\end{minipage}&
\begin{minipage}[c]{5.467cm}\begin{styleTableContents}
\textstyleSourceText{{\textbackslash}author\{...\}}
\end{styleTableContents}
\end{minipage}\\\hline
\begin{minipage}[c]{5.467cm}\begin{styleTableContents}
date
\end{styleTableContents}
\end{minipage}&
\begin{minipage}[c]{5.467cm}\begin{styleTableContents}
paragraph
\end{styleTableContents}
\end{minipage}&
\begin{minipage}[c]{5.467cm}\begin{styleTableContents}
\textstyleSourceText{{\textbackslash}date\{...\}}
\end{styleTableContents}
\end{minipage}\\\hline
\begin{minipage}[c]{5.467cm}\begin{styleTableContents}
abstract title
\end{styleTableContents}
\end{minipage}&
\begin{minipage}[c]{5.467cm}\begin{styleTableContents}
paragraph
\end{styleTableContents}
\end{minipage}&
\begin{minipage}[c]{5.467cm}\begin{styleTableContents}
renews \textstyleSourceText{{\textbackslash}abstractname}
\end{styleTableContents}
\end{minipage}\\\hline
\begin{minipage}[c]{5.467cm}\begin{styleTableContents}
abstract
\end{styleTableContents}
\end{minipage}&
\begin{minipage}[c]{5.467cm}\begin{styleTableContents}
paragraph
\end{styleTableContents}
\end{minipage}&
\begin{minipage}[c]{5.467cm}\begin{styleTableContents}
\textstyleSourceText{abstract} environment
\end{styleTableContents}
\end{minipage}\\\hline
\begin{minipage}[c]{5.467cm}\begin{styleTableContents}
part
\end{styleTableContents}
\end{minipage}&
\begin{minipage}[c]{5.467cm}\begin{styleTableContents}
paragraph
\end{styleTableContents}
\end{minipage}&
\begin{minipage}[c]{5.467cm}\begin{styleTableContents}
\textstyleSourceText{{\textbackslash}part\{...\}}
\end{styleTableContents}
\end{minipage}\\\hline
\begin{minipage}[c]{5.467cm}\begin{stylePxxvii}
Heading 2
\end{stylePxxvii}
\end{minipage}&
\begin{minipage}[c]{5.467cm}\begin{styleTableContents}
paragraph
\end{styleTableContents}
\end{minipage}&
\begin{minipage}[c]{5.467cm}\begin{styleTableContents}
\textstyleSourceText{{\textbackslash}section\{...\}}
\end{styleTableContents}
\end{minipage}\\\hline
\begin{minipage}[c]{5.467cm}\begin{stylePxxvii}
Heading 3
\end{stylePxxvii}
\end{minipage}&
\begin{minipage}[c]{5.467cm}\begin{styleTableContents}
paragraph
\end{styleTableContents}
\end{minipage}&
\begin{minipage}[c]{5.467cm}\begin{styleTableContents}
\textstyleSourceText{{\textbackslash}subsection\{...\}}
\end{styleTableContents}
\end{minipage}\\\hline
\begin{minipage}[c]{5.467cm}\begin{stylePxxvii}
Heading 4
\end{stylePxxvii}
\end{minipage}&
\begin{minipage}[c]{5.467cm}\begin{styleTableContents}
paragraph
\end{styleTableContents}
\end{minipage}&
\begin{minipage}[c]{5.467cm}\begin{styleTableContents}
\textstyleSourceText{{\textbackslash}subsubsection\{...\}}
\end{styleTableContents}
\end{minipage}\\\hline
\begin{minipage}[c]{5.467cm}\begin{stylePxxvii}
Heading 5
\end{stylePxxvii}
\end{minipage}&
\begin{minipage}[c]{5.467cm}\begin{styleTableContents}
paragraph
\end{styleTableContents}
\end{minipage}&
\begin{minipage}[c]{5.467cm}\begin{styleTableContents}
\textstyleSourceText{{\textbackslash}paragraph\{...\}}
\end{styleTableContents}
\end{minipage}\\\hline
\begin{minipage}[c]{5.467cm}\begin{stylePxxvii}
Heading 6
\end{stylePxxvii}
\end{minipage}&
\begin{minipage}[c]{5.467cm}\begin{styleTableContents}
paragraph
\end{styleTableContents}
\end{minipage}&
\begin{minipage}[c]{5.467cm}\begin{styleTableContents}
\textstyleSourceText{{\textbackslash}subparagraph\{...\}}
\end{styleTableContents}
\end{minipage}\\\hline
\begin{minipage}[c]{5.467cm}\begin{styleTableContents}
flushleft
\end{styleTableContents}
\end{minipage}&
\begin{minipage}[c]{5.467cm}\begin{styleTableContents}
paragraph
\end{styleTableContents}
\end{minipage}&
\begin{minipage}[c]{5.467cm}\begin{styleTableContents}
\textstyleSourceText{flushleft} environment
\end{styleTableContents}
\end{minipage}\\\hline
\begin{minipage}[c]{5.467cm}\begin{styleTableContents}
flushright
\end{styleTableContents}
\end{minipage}&
\begin{minipage}[c]{5.467cm}\begin{styleTableContents}
paragraph
\end{styleTableContents}
\end{minipage}&
\begin{minipage}[c]{5.467cm}\begin{styleTableContents}
\textstyleSourceText{flushright} environment
\end{styleTableContents}
\end{minipage}\\\hline
\begin{minipage}[c]{5.467cm}\begin{styleTableContents}
center
\end{styleTableContents}
\end{minipage}&
\begin{minipage}[c]{5.467cm}\begin{styleTableContents}
paragraph
\end{styleTableContents}
\end{minipage}&
\begin{minipage}[c]{5.467cm}\begin{styleTableContents}
\textstyleSourceText{center} environment
\end{styleTableContents}
\end{minipage}\\\hline
\begin{minipage}[c]{5.467cm}\begin{styleTableContents}
verse
\end{styleTableContents}
\end{minipage}&
\begin{minipage}[c]{5.467cm}\begin{styleTableContents}
paragraph
\end{styleTableContents}
\end{minipage}&
\begin{minipage}[c]{5.467cm}\begin{styleTableContents}
\textstyleSourceText{verse} environment
\end{styleTableContents}
\end{minipage}\\\hline
\begin{minipage}[c]{5.467cm}\begin{styleTableContents}
quote
\end{styleTableContents}
\end{minipage}&
\begin{minipage}[c]{5.467cm}\begin{styleTableContents}
paragraph
\end{styleTableContents}
\end{minipage}&
\begin{minipage}[c]{5.467cm}\begin{styleTableContents}
\textstyleSourceText{quote} environment
\end{styleTableContents}
\end{minipage}\\\hline
\begin{minipage}[c]{5.467cm}\begin{styleTableContents}
quotation
\end{styleTableContents}
\end{minipage}&
\begin{minipage}[c]{5.467cm}\begin{styleTableContents}
paragraph
\end{styleTableContents}
\end{minipage}&
\begin{minipage}[c]{5.467cm}\begin{styleTableContents}
\textstyleSourceText{quotation} environment
\end{styleTableContents}
\end{minipage}\\\hline
\begin{minipage}[c]{5.467cm}\begin{stylePxxvii}
Preformatted text
\end{stylePxxvii}
\end{minipage}&
\begin{minipage}[c]{5.467cm}\begin{styleTableContents}
paragraph
\end{styleTableContents}
\end{minipage}&
\begin{minipage}[c]{5.467cm}\begin{styleTableContents}
\textstyleSourceText{verbatim} environment\footnotemark{}
\end{styleTableContents}
\end{minipage}\\\hline
\begin{minipage}[c]{5.467cm}\begin{styleTableContents}
theorem
\end{styleTableContents}
\end{minipage}&
\begin{minipage}[c]{5.467cm}\begin{styleTableContents}
paragraph
\end{styleTableContents}
\end{minipage}&
\begin{minipage}[c]{5.467cm}\begin{styleTableContents}
\textstyleSourceText{theorem} environment
\end{styleTableContents}
\end{minipage}\\\hline
\begin{minipage}[c]{5.467cm}\begin{styleTableContents}
itemize
\end{styleTableContents}
\end{minipage}&
\begin{minipage}[c]{5.467cm}\begin{styleTableContents}
paragraph
\end{styleTableContents}
\end{minipage}&
\begin{minipage}[c]{5.467cm}\begin{styleTableContents}
\textstyleSourceText{itemize} list
\end{styleTableContents}
\end{minipage}\\\hline
\begin{minipage}[c]{5.467cm}\begin{styleTableContents}
enumerate
\end{styleTableContents}
\end{minipage}&
\begin{minipage}[c]{5.467cm}\begin{styleTableContents}
paragraph
\end{styleTableContents}
\end{minipage}&
\begin{minipage}[c]{5.467cm}\begin{styleTableContents}
\textstyleSourceText{enurerate} list
\end{styleTableContents}
\end{minipage}\\\hline
\begin{minipage}[c]{5.467cm}\begin{styleTableContents}
List Heading
\end{styleTableContents}
\end{minipage}&
\begin{minipage}[c]{5.467cm}\begin{styleTableContents}
paragraph
\end{styleTableContents}
\end{minipage}&
\begin{minipage}[c]{5.467cm}\begin{styleTableContents}
\textstyleSourceText{description} list (item label)
\end{styleTableContents}
\end{minipage}\\\hline
\begin{minipage}[c]{5.467cm}\begin{styleTableContents}
List Contents
\end{styleTableContents}
\end{minipage}&
\begin{minipage}[c]{5.467cm}\begin{styleTableContents}
paragraph
\end{styleTableContents}
\end{minipage}&
\begin{minipage}[c]{5.467cm}\begin{styleTableContents}
\textstyleSourceText{description} list (item text)
\end{styleTableContents}
\end{minipage}\\\hline
\begin{minipage}[c]{5.467cm}\begin{styleTableContents}
verb
\end{styleTableContents}
\end{minipage}&
\begin{minipage}[c]{5.467cm}\begin{styleTableContents}
text
\end{styleTableContents}
\end{minipage}&
\begin{minipage}[c]{5.467cm}\begin{styleTableContents}
\textstyleSourceText{{\textbackslash}verb{\textbar}...{\textbar}}
\end{styleTableContents}
\end{minipage}\\\hline
\begin{minipage}[c]{5.467cm}\begin{stylePxxvii}
Emphasis
\end{stylePxxvii}
\end{minipage}&
\begin{minipage}[c]{5.467cm}\begin{styleTableContents}
text
\end{styleTableContents}
\end{minipage}&
\begin{minipage}[c]{5.467cm}\begin{styleTableContents}
\textstyleSourceText{{\textbackslash}emph\{...\}}
\end{styleTableContents}
\end{minipage}\\\hline
\begin{minipage}[c]{5.467cm}\begin{stylePxxvii}
Strong Emphasis
\end{stylePxxvii}
\end{minipage}&
\begin{minipage}[c]{5.467cm}\begin{styleTableContents}
text
\end{styleTableContents}
\end{minipage}&
\begin{minipage}[c]{5.467cm}\begin{styleTableContents}
\textstyleSourceText{{\textbackslash}textbf\{...\}}
\end{styleTableContents}
\end{minipage}\\\hline
\begin{minipage}[c]{5.467cm}\begin{styleTableContents}
textrm
\end{styleTableContents}
\end{minipage}&
\begin{minipage}[c]{5.467cm}\begin{styleTableContents}
text
\end{styleTableContents}
\end{minipage}&
\begin{minipage}[c]{5.467cm}\begin{styleTableContents}
\textstyleSourceText{{\textbackslash}textrm\{...\}}
\end{styleTableContents}
\end{minipage}\\\hline
\begin{minipage}[c]{5.467cm}\begin{styleTableContents}
textsf
\end{styleTableContents}
\end{minipage}&
\begin{minipage}[c]{5.467cm}\begin{styleTableContents}
text
\end{styleTableContents}
\end{minipage}&
\begin{minipage}[c]{5.467cm}\begin{styleTableContents}
\textstyleSourceText{{\textbackslash}textsf\{...\}}
\end{styleTableContents}
\end{minipage}\\\hline
\begin{minipage}[c]{5.467cm}\begin{styleTableContents}
texttt
\end{styleTableContents}
\end{minipage}&
\begin{minipage}[c]{5.467cm}\begin{styleTableContents}
text
\end{styleTableContents}
\end{minipage}&
\begin{minipage}[c]{5.467cm}\begin{styleTableContents}
\textstyleSourceText{{\textbackslash}texttt\{...\}}
\end{styleTableContents}
\end{minipage}\\\hline
\begin{minipage}[c]{5.467cm}\begin{styleTableContents}
textup
\end{styleTableContents}
\end{minipage}&
\begin{minipage}[c]{5.467cm}\begin{styleTableContents}
text
\end{styleTableContents}
\end{minipage}&
\begin{minipage}[c]{5.467cm}\begin{styleTableContents}
\textstyleSourceText{{\textbackslash}textup\{...\}}
\end{styleTableContents}
\end{minipage}\\\hline
\begin{minipage}[c]{5.467cm}\begin{styleTableContents}
textsl
\end{styleTableContents}
\end{minipage}&
\begin{minipage}[c]{5.467cm}\begin{styleTableContents}
text
\end{styleTableContents}
\end{minipage}&
\begin{minipage}[c]{5.467cm}\begin{styleTableContents}
\textstyleSourceText{{\textbackslash}textsl\{...\}}
\end{styleTableContents}
\end{minipage}\\\hline
\begin{minipage}[c]{5.467cm}\begin{styleTableContents}
textit
\end{styleTableContents}
\end{minipage}&
\begin{minipage}[c]{5.467cm}\begin{styleTableContents}
text
\end{styleTableContents}
\end{minipage}&
\begin{minipage}[c]{5.467cm}\begin{styleTableContents}
\textstyleSourceText{{\textbackslash}textit\{...\}}
\end{styleTableContents}
\end{minipage}\\\hline
\begin{minipage}[c]{5.467cm}\begin{styleTableContents}
textsc
\end{styleTableContents}
\end{minipage}&
\begin{minipage}[c]{5.467cm}\begin{styleTableContents}
text
\end{styleTableContents}
\end{minipage}&
\begin{minipage}[c]{5.467cm}\begin{styleTableContents}
\textstyleSourceText{{\textbackslash}textsc\{...\}}
\end{styleTableContents}
\end{minipage}\\\hline
\begin{minipage}[c]{5.467cm}\begin{styleTableContents}
textmd
\end{styleTableContents}
\end{minipage}&
\begin{minipage}[c]{5.467cm}\begin{styleTableContents}
text
\end{styleTableContents}
\end{minipage}&
\begin{minipage}[c]{5.467cm}\begin{styleTableContents}
\textstyleSourceText{{\textbackslash}textmd\{...\}}
\end{styleTableContents}
\end{minipage}\\\hline
\begin{minipage}[c]{5.467cm}\begin{styleTableContents}
textbf
\end{styleTableContents}
\end{minipage}&
\begin{minipage}[c]{5.467cm}\begin{styleTableContents}
text
\end{styleTableContents}
\end{minipage}&
\begin{minipage}[c]{5.467cm}\begin{styleTableContents}
\textstyleSourceText{{\textbackslash}textbf\{...\}}
\end{styleTableContents}
\end{minipage}\\\hline
\begin{minipage}[c]{5.467cm}\begin{styleTableContents}
tiny
\end{styleTableContents}
\end{minipage}&
\begin{minipage}[c]{5.467cm}\begin{styleTableContents}
text
\end{styleTableContents}
\end{minipage}&
\begin{minipage}[c]{5.467cm}\begin{styleTableContents}
\textstyleSourceText{\{{\textbackslash}tiny ...\}}
\end{styleTableContents}
\end{minipage}\\\hline
\begin{minipage}[c]{5.467cm}\begin{styleTableContents}
scriptsize
\end{styleTableContents}
\end{minipage}&
\begin{minipage}[c]{5.467cm}\begin{styleTableContents}
text
\end{styleTableContents}
\end{minipage}&
\begin{minipage}[c]{5.467cm}\begin{styleTableContents}
\textstyleSourceText{\{{\textbackslash}sciptsize ...\}}
\end{styleTableContents}
\end{minipage}\\\hline
\begin{minipage}[c]{5.467cm}\begin{styleTableContents}
footnotesize
\end{styleTableContents}
\end{minipage}&
\begin{minipage}[c]{5.467cm}\begin{styleTableContents}
text
\end{styleTableContents}
\end{minipage}&
\begin{minipage}[c]{5.467cm}\begin{styleTableContents}
\textstyleSourceText{\{{\textbackslash}footnotesize ...\}}
\end{styleTableContents}
\end{minipage}\\\hline
\begin{minipage}[c]{5.467cm}\begin{styleTableContents}
small
\end{styleTableContents}
\end{minipage}&
\begin{minipage}[c]{5.467cm}\begin{styleTableContents}
text
\end{styleTableContents}
\end{minipage}&
\begin{minipage}[c]{5.467cm}\begin{styleTableContents}
\textstyleSourceText{\{{\textbackslash}small ...\}}
\end{styleTableContents}
\end{minipage}\\\hline
\begin{minipage}[c]{5.467cm}\begin{styleTableContents}
normalsize
\end{styleTableContents}
\end{minipage}&
\begin{minipage}[c]{5.467cm}\begin{styleTableContents}
text
\end{styleTableContents}
\end{minipage}&
\begin{minipage}[c]{5.467cm}\begin{styleTableContents}
\textstyleSourceText{\{{\textbackslash}normalsize ...\}}
\end{styleTableContents}
\end{minipage}\\\hline
\begin{minipage}[c]{5.467cm}\begin{styleTableContents}
large
\end{styleTableContents}
\end{minipage}&
\begin{minipage}[c]{5.467cm}\begin{styleTableContents}
text
\end{styleTableContents}
\end{minipage}&
\begin{minipage}[c]{5.467cm}\begin{styleTableContents}
\textstyleSourceText{\{{\textbackslash}large ...\}}
\end{styleTableContents}
\end{minipage}\\\hline
\begin{minipage}[c]{5.467cm}\begin{styleTableContents}
Large
\end{styleTableContents}
\end{minipage}&
\begin{minipage}[c]{5.467cm}\begin{styleTableContents}
text
\end{styleTableContents}
\end{minipage}&
\begin{minipage}[c]{5.467cm}\begin{styleTableContents}
\textstyleSourceText{\{{\textbackslash}Large ...\}}
\end{styleTableContents}
\end{minipage}\\\hline
\begin{minipage}[c]{5.467cm}\begin{styleTableContents}
LARGE
\end{styleTableContents}
\end{minipage}&
\begin{minipage}[c]{5.467cm}\begin{styleTableContents}
text
\end{styleTableContents}
\end{minipage}&
\begin{minipage}[c]{5.467cm}\begin{styleTableContents}
\textstyleSourceText{\{{\textbackslash}LARGE ...\}}
\end{styleTableContents}
\end{minipage}\\\hline
\begin{minipage}[c]{5.467cm}\begin{styleTableContents}
huge
\end{styleTableContents}
\end{minipage}&
\begin{minipage}[c]{5.467cm}\begin{styleTableContents}
text
\end{styleTableContents}
\end{minipage}&
\begin{minipage}[c]{5.467cm}\begin{styleTableContents}
\textstyleSourceText{\{{\textbackslash}huge ...\}}
\end{styleTableContents}
\end{minipage}\\\hline
\begin{minipage}[c]{5.467cm}\begin{styleTableContents}
Huge
\end{styleTableContents}
\end{minipage}&
\begin{minipage}[c]{5.467cm}\begin{styleTableContents}
text
\end{styleTableContents}
\end{minipage}&
\begin{minipage}[c]{5.467cm}\begin{styleTableContents}
\textstyleSourceText{\{{\textbackslash}Huge ...\}}
\end{styleTableContents}
\end{minipage}\\\hline
\end{longtable}
\addtocounter{footnote}{-3}
\stepcounter{footnote}\footnotetext{The use of italics in this table
indicates styles that are predefined in OOo. The names of these styles
will be localized if you use a non{}-english version of OOo.}
\stepcounter{footnote}\footnotetext{Also
\textstyleSourceText{{\textbackslash}maketitle} is added at the end of
a sequence of Title, author and date.}
\stepcounter{footnote}\footnotetext{Only characters available in the
inputenc are accepted. Other characters are converted to question marks
and other content is discarded, eg. footnotes.}
\begin{styleTextbody}
If you use these styles and uses the configuration file
\textstyleSourceText{article.xml} when you convert your document with
Writer2LaTeX, you will get a result that resembles a handwritten LaTeX
file. Note that hard formatting and any other styles will be ignored.
\end{styleTextbody}

\subsubsection{Roundtrip editing}
\begin{styleTextbody}
Writer2LaTeX does not provide a filter, that converts LaTeX files back
into OOo Writer format. This is however possible with Eitan M.
Gurari{\textquotesingle}s TeX4ht system
(\href{http://www.cse.ohio-state.edu/~gurari/TeX4ht/mn.html}{http://www.cse.ohio{}-state.edu/\~{}gurari/TeX4ht/mn.html}).
If you use Writer2LaTeX (with \textstyleSourceText{article.xml})
together with TeX4ht{\textquotesingle}s OOo mode
(\textstyleSourceText{oolatex}), simple roundtrip edition LaTeX
\ding{214} OOo Writer is supported. Beware of information loss if you
do this {--} do not use this roundtrip for existing LaTeX or Writer
documents!
\end{styleTextbody}

\begin{styleTextbody}
As a genereal rule, you should save your document in the native OOo
Writer format and convert to LaTeX when you are finished (or want to
see the result).
\end{styleTextbody}

\section{Using Writer2xhtml and Calc2xhtml}
\begin{styleTextbody}
Writer2xhtml is producing standards compliant XHTML files, in particular
it can be used to put math on the web using the XHTML + MathML
combination. Thus Writer2xhtml can convert into any of these XHTML
variants:
\end{styleTextbody}

\begin{listLxviiileveli}
\item 
\begin{stylePxxviii}
XHTML 1.0 strict, which follows the guidelines for HTML compatibility,
so that the output should be viewable with any browser that supports
HTML 4.
\end{stylePxxviii}
\item 
\begin{stylePxxviii}
XHTML 1.1 + MathML 2.0, which currently is viewable with the Mozilla and
Amaya browsers only.
\end{stylePxxviii}
\item 
\begin{stylePxxviii}
XHTML 1.1 + MathML 2.0 using \href{http://www.w3.org/Math/XSL/}{XSL
transformations from the W3C Math Working Group} to make the file
viewable also in some browsers that needs a plugin to display MathML,
eg. Internet Explorer with MathPlayer plugin.
\end{stylePxxviii}

\begin{stylePxxviii}
This is how W3C{\textquotesingle}s Math Working Group recommends to put
''math on the web''.
\end{stylePxxviii}
\end{listLxviiileveli}
\begin{styleTextbody}
Note that the default file extension and the recommended MIME types
varies with the output format:
\end{styleTextbody}

\begin{longtable}[c]{|p{5.467cm}|p{5.467cm}|p{5.467cm}|}
\hline
\begin{minipage}[c]{5.467cm}\begin{styleTableHeading}
Output format
\end{styleTableHeading}
\end{minipage}&
\begin{minipage}[c]{5.467cm}\begin{styleTableHeading}
Default file extenstion
\end{styleTableHeading}
\end{minipage}&
\begin{minipage}[c]{5.467cm}\begin{styleTableHeading}
MIME type
\end{styleTableHeading}
\end{minipage}\\\hline
\endhead
\begin{minipage}[c]{5.467cm}\begin{styleTableContents}
XHTML 1.0
\end{styleTableContents}
\end{minipage}&
\begin{minipage}[c]{5.467cm}\begin{styleTableContents}
\textstyleSourceText{.html}
\end{styleTableContents}
\end{minipage}&
\begin{minipage}[c]{5.467cm}\begin{styleTableContents}
\textstyleSourceText{text/html}
\end{styleTableContents}
\end{minipage}\\\hline
\begin{minipage}[c]{5.467cm}\begin{styleTableContents}
XHTML 1.1 + MathML 2.0
\end{styleTableContents}
\end{minipage}&
\begin{minipage}[c]{5.467cm}\begin{styleTableContents}
\textstyleSourceText{.xhtml}
\end{styleTableContents}
\end{minipage}&
\begin{minipage}[c]{5.467cm}\begin{styleTableContents}
\textstyleSourceText{application/xhtml+xml}
\end{styleTableContents}
\end{minipage}\\\hline
\begin{minipage}[c]{5.467cm}\begin{styleTableContents}
XHTML 1.1 + MathML 2.0 (with xsl transformation)
\end{styleTableContents}
\end{minipage}&
\begin{minipage}[c]{5.467cm}\begin{styleTableContents}
\textstyleSourceText{.xml}
\end{styleTableContents}
\end{minipage}&
\begin{minipage}[c]{5.467cm}\begin{styleTableContents}
\textstyleSourceText{application/xml}
\end{styleTableContents}
\end{minipage}\\\hline
\end{longtable}
\begin{styleTextbody}
Writer2xhtml is quite flexible; in particular with respect to the
handling of formatting:
\end{styleTextbody}

\begin{listLxixleveli}
\item 
\begin{stylePxxix}
You can let Writer2xhtml convert the style information in the source
document and thus get an xhtml document that has the same general
appearance as the original, but with an online look and feel.
\end{stylePxxix}
\item 
\begin{stylePxxix}
You can use your own style sheet and let Writer2xhtml convert the
content only. You can map styles in OOo to xhtml elements and css
classes, see sections 4.3 and 4.4
\end{stylePxxix}
\end{listLxixleveli}
\begin{styleTextbody}
Calc2xhtml is a companion to Writer2xhtml that produces XHTML 1.0 strict
from your Calc documents.
\end{styleTextbody}

\subsection{Converting to XHTML from the command line}
\begin{styleTextbody}
To convert a file to XHTML use the command line
\end{styleTextbody}

\begin{stylePreformattedText}
w2l {}-xhtml{\textbar}{}-xhtml+mathml{\textbar}{}-xhtml+mathml+xsl
[{}-config {\textless}configfile{\textgreater}] {\textless}document to
convert{\textgreater} [{\textless}output path and/or file
name{\textgreater}]
\end{stylePreformattedText}

\begin{styleTextbody}
The parts in square brackets are optional.
\end{styleTextbody}

\begin{styleTextbody}
This will produce an XHTML file with the specified name. If no output
file is specified, Writer2xhtml will use the same name as the original
document, but a different file extension.
\end{styleTextbody}

\begin{styleTextbody}
The option \textstyleUserEntry{{}-xhtml+mathml} is used to produce XHTML
1.1 + MathML 2.0, the option \textstyleUserEntry{{}-xhtml+mathml+xsl}
produces the variant using XSL transformations.
\end{styleTextbody}

\begin{styleTextbody}
Examples:
\end{styleTextbody}

\begin{stylePreformattedText}
w2l {}-xhtml+mathml+xsl mydocument.sxw
\end{stylePreformattedText}

\begin{styleTextbody}
or
\end{styleTextbody}

\begin{stylePreformattedText}
w2l {}-xhtml {}-config myconfig.xml mydocument.sxw
\end{stylePreformattedText}

\begin{styleTextbody}
If you specify the \textstyleTeletype{{}-config} option, Writer2xhtml
will load this configuration file before converting your document. You
can read more about configuration in section 4.3.
\end{styleTextbody}

\begin{styleTextbody}
The script \textstyleSourceText{w2l} also provides a shorthand notation
to use the sample configuration file included in
\textstyleSourceText{writer2latex04.zip}. The command line is
\end{styleTextbody}

\begin{stylePreformattedText}
w2l {}-cleanxhtml {\textless}writer document to convert{\textgreater}
[{\textless}output path and/or file name{\textgreater}]
\end{stylePreformattedText}

\begin{styleTextbody}
This configuration file produces a ''clean'' xhtml file (see section
4.4), for example:
\end{styleTextbody}

\begin{stylePreformattedText}
w2l {}-cleanxhtml mydocument.sxw mypath/myoutputdoc.html
\end{stylePreformattedText}

\begin{styleTextbody}
It is recommended that you extend \textstyleSourceText{w2l} /
\textstyleSourceText{w2l.bat} to support your own configuration files.
\end{styleTextbody}

\subsection{Using Writer2xhtml as an export filter}
\begin{styleTextbody}
If you choose \textbf{File {--} Export} in Writer you should be able to
choose \textbf{XHTML 1.0 strict}, \textbf{XHTML 1.1 + MathML 2.0} or
\textbf{XHTML 1.1 + MathML 2.0 (xsl)} as file type. Using Calc2xhtml as
an export filter is not yet supported.
\end{styleTextbody}

\begin{styleTextbody}
\textbf{Note:} You have to use the export menu because Writer2xhtml does
not provide an import filter for XHTML. You should always save in the
native format of OOo as well!
\end{styleTextbody}

\begin{styleTextbody}
Note: Currently embedded graphics are not converted when Writer2xhtml is
used as an export filter. Also splitting at headings/sheets only works
from the command line. This is because of an issue with the xmerge
framework. A fix for this is planned for a later version of
Writer2xhtml.
\end{styleTextbody}

\subsection{\label{ref:xhtmlconfig}Configuration}
\begin{styleTextbody}
XHTML export can be configured with a configuration file. The
configuration is read from several sources:
\end{styleTextbody}

\begin{listLxleveli}
\item 
\begin{stylePxix}
First Writer2xhtml/Calc2xhtml reads the file
\textstyleTeletype{writer2latex.xml} in the same directory as
\textstyleTeletype{writer2latex.jar}. This file is supposed to contain
installation{}-wide configuration.
\end{stylePxix}
\item 
\begin{stylePxix}
Then it reads the file \textstyleSourceText{writer2latex.xml} in your
home directory (unix, eg. \textstyleSourceText{/home/username}) or user
profile (windows, eg. \textstyleSourceText{c:{\textbackslash}documents
and settings{\textbackslash}username}). This file is supposed to
contain user{}-specific configuration. The installation{}-wide
configuration may specify, that this file should be generated
automatically.
\end{stylePxix}
\item 
\begin{stylePxix}
Finally the configuration file you specify on the command line is read.
\end{stylePxix}
\end{listLxleveli}
\begin{styleTextbody}
The configuration file is an xml file, here are the default contents:
\end{styleTextbody}

\begin{stylePreformattedText}
{\textless}?xml version={\textquotedbl}1.0{\textquotedbl}
encoding={\textquotedbl}UTF{}-8{\textquotedbl}?{\textgreater}
\end{stylePreformattedText}

\begin{stylePreformattedText}
{\textless}config{\textgreater}
\end{stylePreformattedText}

\begin{stylePreformattedText}
 {\textless}option name=''create\_user\_config'' value=''true''
/{\textgreater}
\end{stylePreformattedText}

\begin{stylePreformattedText}
 {\textless}option name={\textquotedbl}xhtml\_no\_doctype{\textquotedbl}
value={\textquotedbl}false{\textquotedbl} /{\textgreater}
\end{stylePreformattedText}

\begin{stylePreformattedText}
 {\textless}option
name={\textquotedbl}xhtml\_custom\_stylesheet{\textquotedbl}
value={\textquotedbl}{\textquotedbl} /{\textgreater}
\end{stylePreformattedText}

\begin{stylePreformattedText}
 {\textless}option
name={\textquotedbl}xhtml\_ignore\_styles{\textquotedbl}
value={\textquotedbl}false{\textquotedbl} /{\textgreater}
\end{stylePreformattedText}

\begin{stylePreformattedText}
 {\textless}option
name={\textquotedbl}xhtml\_use\_dublin\_core{\textquotedbl}
value={\textquotedbl}true{\textquotedbl} /{\textgreater}
\end{stylePreformattedText}

\begin{stylePreformattedText}
 {\textless}option
name={\textquotedbl}xhtml\_convert\_to\_px{\textquotedbl}
value={\textquotedbl}true{\textquotedbl} /{\textgreater}
\end{stylePreformattedText}

\begin{stylePreformattedText}
 {\textless}option name={\textquotedbl}xhtml\_scaling{\textquotedbl}
value={\textquotedbl}100\%{\textquotedbl} /{\textgreater}
\end{stylePreformattedText}

\begin{stylePreformattedText}
 {\textless}option
name={\textquotedbl}xhtml\_column\_scaling{\textquotedbl}
value={\textquotedbl}100\%{\textquotedbl} /{\textgreater}
\end{stylePreformattedText}

\begin{stylePreformattedText}
 {\textless}option
name={\textquotedbl}xhtml\_split\_level{\textquotedbl}
value={\textquotedbl}0{\textquotedbl} /{\textgreater}
\end{stylePreformattedText}

\begin{stylePreformattedText}
 {\textless}option name={\textquotedbl}xhtml\_calc\_split{\textquotedbl}
value={\textquotedbl}false{\textquotedbl} /{\textgreater}
\end{stylePreformattedText}

\begin{stylePreformattedText}
 {\textless}option
name={\textquotedbl}ignore\_hard\_line\_breaks{\textquotedbl}
value={\textquotedbl}false{\textquotedbl} /{\textgreater}
\end{stylePreformattedText}

\begin{stylePreformattedText}
 {\textless}option
name={\textquotedbl}ignore\_empty\_paragraphs{\textquotedbl}
value={\textquotedbl}false{\textquotedbl} /{\textgreater}
\end{stylePreformattedText}

\begin{stylePreformattedText}
 {\textless}option
name={\textquotedbl}ignore\_double\_spaces{\textquotedbl}
value={\textquotedbl}false{\textquotedbl} /{\textgreater}
\end{stylePreformattedText}

\begin{stylePreformattedText}
{\textless}/config{\textgreater}
\end{stylePreformattedText}

\subsubsection{Options}
\begin{listLxiileveli}
\item 
\begin{stylePxxi}
If the option\textstyleUserEntry{ create\_user\_config }if set to
\textstyleUserEntry{true}, the user specific configuration file
mentioned above will be created if it does not exist.
\end{stylePxxi}
\item 
\begin{stylePxxi}
The option \textstyleTeletype{xhtml\_no\_doctype} can have the values
\textstyleSourceText{true} or \textstyleSourceText{false} (default).
When this option is \textstyleSourceText{true}, Writer2xhtml will not
include the !DOCTYPE declaration in the converted document. The
!DOCTYPE is required for a valid xhtml document; this option should
only be used if you need to process the document further.
\end{stylePxxi}
\item 
\begin{stylePxxi}
The option \textstyleUserEntry{xhtml\_custom\_stylesheet} is used to
specify an URL to your own, external stylesheet. If the value is empty
or the option is not specified, no external stylesheet will be used.
\end{stylePxxi}
\item 
\begin{stylePxxi}
The option \textstyleUserEntry{xhtml\_ignore\_styles }is used to specify
if formatting should be exported. If the value is true, no style
information will be exported (in this case you should specify a custom
style sheet!).
\end{stylePxxi}
\item 
\begin{stylePxxi}
The option \textstyleUserEntry{xhtml\_use\_dublin\_core }is used to
specify if Dublin Core Meta data should be exported (the format will be
as specified in
\href{http://dublincore.org/documents/dcq-html/}{http://dublincore.org/documents/dcq{}-html/}).
If the value is \textstyleUserEntry{false}, it will not be exported.
\end{stylePxxi}
\item 
\begin{stylePxxi}
The option \textstyleTeletype{xhtml\_convert\_to\_px} can have the
values \textstyleSourceText{true} (default) or
\textstyleSourceText{false}. When this option is
\textstyleSourceText{true}, Writer2xhtml will convert all units to
\textstyleTeletype{px}, otherwise the original units are used. The
resolution is assumed to be 96ppi, you can change this with the
\textstyleTeletype{xhtml\_scaling} option. Eg. a scaling
\textstyleTeletype{75\%} will change the resolution to 72ppi.
\end{stylePxxi}
\item 
\begin{stylePxxi}
The option  \textstyleUserEntry{xhtml\_scaling }is used to specify a
scaling of all formatting, ie. to get a different text size than the
original document. The value must be a percentage.
\end{stylePxxi}
\item 
\begin{stylePxxi}
The option  \textstyleUserEntry{xhtml\_column\_scaling }is used to
specify an additional scaling for table colums. The value must be a
percentage.
\end{stylePxxi}
\item 
\begin{stylePxxi}
The option  \textstyleUserEntry{xhtml\_split\_level }is used to specify
that the Writer documents should be split in several documents and the
outline level at which the splitting should happen (the default 0 means
no split). This is convenient for long documents. Each output document
will get a simple navigation panel in the header and the footer.
\end{stylePxxi}
\item 
\begin{stylePxxi}
The option  \textstyleUserEntry{xhtml\_calc\_split }is used to specify
that the Calc documents should be split in several documents, one for
each sheet. This is convenient for large spreadsheets. Each output
document will get a simple navigation panel in the header and the
footer.
\end{stylePxxi}
\item 
\begin{stylePxxi}
The option \textstyleSourceText{ignore\_double\_spaces} can have the
values \textstyleSourceText{true} (default) or
\textstyleSourceText{false}. Setting the option to
\textstyleSourceText{true} will instruct Writer2xhtml to ignore double
spaces, otherwise they are converted to non{}-breaking spaces.
\end{stylePxxi}
\end{listLxiileveli}
\begin{listLxviileveli}
\item 
\begin{stylePxxvi}
The option \textstyleSourceText{ignore\_empty\_paragraphs} can have the
values \textstyleSourceText{true} (default) or
\textstyleSourceText{false}. Setting the option to
\textstyleSourceText{true} will instruct Writer2xhtml to ignore empty
paragraphs..
\end{stylePxxvi}
\item 
\begin{stylePxxvi}
The option \textstyleSourceText{ignore\_hard\_line\_breaks} can have the
values \textstyleSourceText{true} or \textstyleSourceText{false}
(default).  Setting the option to \textstyleSourceText{true} will
instruct Writer2xhtml to ignore hard line breaks (shift{}-Enter).
\end{stylePxxvi}
\end{listLxviileveli}
\subsubsection{Style maps}
\begin{styleTextbody}
In addition to the options, you can specify that certain styles in
Writer should be mapped to specific XHTML elements and CSS style
classes. Here are some examples showing how to use some of the
built{}-in Writer styles to create XHTML elements:
\end{styleTextbody}

\begin{stylePreformattedText}
{\textless}?xml version={\textquotedbl}1.0{\textquotedbl}
encoding={\textquotedbl}UTF{}-8{\textquotedbl}?{\textgreater}
\end{stylePreformattedText}

\begin{stylePreformattedText}
{\textless}config{\textgreater}
\end{stylePreformattedText}

\begin{stylePreformattedText}
 {\textless}!{}-{}- map OOo paragraph styles to xhtml elements
{}-{}-{\textgreater}
\end{stylePreformattedText}

\begin{stylePreformattedText}
 {\textless}xhtml{}-style{}-map name={\textquotedbl}Text
body{\textquotedbl} class={\textquotedbl}paragraph{\textquotedbl}  
\end{stylePreformattedText}

\begin{stylePreformattedText}
 element={\textquotedbl}p{\textquotedbl}
css={\textquotedbl}(none){\textquotedbl} /{\textgreater}  
\end{stylePreformattedText}

\begin{stylePreformattedText}
 {\textless}xhtml{}-style{}-map
name={\textquotedbl}Sender{\textquotedbl}
class={\textquotedbl}paragraph{\textquotedbl}
\end{stylePreformattedText}

\begin{stylePreformattedText}
 element={\textquotedbl}address{\textquotedbl}
css={\textquotedbl}(none){\textquotedbl} /{\textgreater}
\end{stylePreformattedText}

\begin{stylePreformattedText}
 {\textless}xhtml{}-style{}-map
name={\textquotedbl}Quotations{\textquotedbl}
class={\textquotedbl}paragraph{\textquotedbl}
\end{stylePreformattedText}

\begin{stylePreformattedText}
 block{}-element={\textquotedbl}blockquote{\textquotedbl}
block{}-css={\textquotedbl}(none){\textquotedbl}
\end{stylePreformattedText}

\begin{stylePreformattedText}
 element={\textquotedbl}p{\textquotedbl}
css={\textquotedbl}(none){\textquotedbl} /{\textgreater}
\end{stylePreformattedText}


\bigskip

\begin{stylePreformattedText}
 {\textless}!{}-{}- map OOo text styles to xhtml elements
{}-{}-{\textgreater}
\end{stylePreformattedText}

\begin{stylePreformattedText}
 {\textless}xhtml{}-style{}-map
name={\textquotedbl}Citation{\textquotedbl}
class={\textquotedbl}text{\textquotedbl}
\end{stylePreformattedText}

\begin{stylePreformattedText}
 element={\textquotedbl}cite{\textquotedbl}
css={\textquotedbl}(none){\textquotedbl} /{\textgreater}
\end{stylePreformattedText}

\begin{stylePreformattedText}
 {\textless}xhtml{}-style{}-map
name={\textquotedbl}Emphasis{\textquotedbl}
class={\textquotedbl}text{\textquotedbl}
\end{stylePreformattedText}

\begin{stylePreformattedText}
 element={\textquotedbl}em{\textquotedbl}
css={\textquotedbl}(none){\textquotedbl} /{\textgreater}
\end{stylePreformattedText}

\begin{stylePreformattedText}
 
\end{stylePreformattedText}

\begin{stylePreformattedText}
 {\textless}!{}-{}- map hard formatting attributes to xhtml elements
{}-{}-{\textgreater}
\end{stylePreformattedText}

\begin{stylePreformattedText}
 {\textless}xhtml{}-style{}-map name={\textquotedbl}bold{\textquotedbl}
class={\textquotedbl}attribute{\textquotedbl}
\end{stylePreformattedText}

\begin{stylePreformattedText}
 element={\textquotedbl}b{\textquotedbl}
css={\textquotedbl}(none){\textquotedbl} /{\textgreater}
\end{stylePreformattedText}

\begin{stylePreformattedText}
 {\textless}xhtml{}-style{}-map
name={\textquotedbl}italics{\textquotedbl}
class={\textquotedbl}attribute{\textquotedbl}
\end{stylePreformattedText}

\begin{stylePreformattedText}
 element={\textquotedbl}i{\textquotedbl}
css={\textquotedbl}(none){\textquotedbl} /{\textgreater}
\end{stylePreformattedText}

\begin{stylePreformattedText}
{\textless}/config{\textgreater}
\end{stylePreformattedText}

\begin{styleTextbody}
An extended version of this is distributed with Writer2LaTeX, please see
the file \textstyleSourceText{cleanxhtml.xml}.
\end{styleTextbody}

\begin{styleTextbody}
The attributes of the \textstyleUserEntry{xhtml{}-style{}-map} element
are used as follows:
\end{styleTextbody}

\begin{listLxxleveli}
\item 
\begin{stylePxxx}
\textstyleUserEntry{name} specifies the name of the Writer style.
\end{stylePxxx}
\item 
\begin{stylePxxx}
\textstyleUserEntry{class} specifies the styles class in Writer; this
can either be \textstyleUserEntry{text},
\textstyleUserEntry{paragraph}, \textstyleUserEntry{frame},
\textstyleUserEntry{list} or \textstyleUserEntry{attribute}. The last
value does not specify a real style, but refers to hard formatting
attributes. The possible names in this case are
\textstyleUserEntry{bold}, \textstyleUserEntry{italics},
\textstyleUserEntry{fixed} (for fixed pitch fonts),
\textstyleUserEntry{superscript} and \textstyleUserEntry{subscript}.
\end{stylePxxx}
\item 
\begin{stylePxxx}
\textstyleUserEntry{element} specifies the XHTML element to use when
converting this style. This is not used for frame and list styles.
\end{stylePxxx}
\item 
\begin{stylePxxx}
\textstyleUserEntry{css} specifies the CSS style class to use when
converting this style. If it is not specified or the value is
\textstyleUserEntry{``(none)''}, no CSS class will be used.
\end{stylePxxx}
\item 
\begin{stylePxxx}
\textstyleUserEntry{block{}-element} only has effect for paragraph
styles. It is used to specify a block XHTML element, that should
surround several exported paragraphs with this style.
\end{stylePxxx}
\item 
\begin{stylePxxx}
\textstyleUserEntry{block{}-css} specifies the CSS style class to be
used for this block element. If it is not specified or the value is
\textstyleUserEntry{``(none)''}, no CSS class will be used.
\end{stylePxxx}
\end{listLxxleveli}
\begin{styleTextbody}
For example the rules above produces code like this:
\end{styleTextbody}

\begin{stylePreformattedText}
{\textless}p{\textgreater}This paragraph is Text
body{\textless}/p{\textgreater}
\end{stylePreformattedText}

\begin{stylePreformattedText}
{\textless}address{\textgreater}This paragraph is
Sender{\textless}/address{\textgreater}
\end{stylePreformattedText}

\begin{stylePreformattedText}
{\textless}blockquote{\textgreater}
\end{stylePreformattedText}

\begin{stylePreformattedText}
 {\textless}p{\textgreater}This paragraph is
Quotations{\textless}/p{\textgreater}
\end{stylePreformattedText}

\begin{stylePreformattedText}
 {\textless}p{\textgreater}This paragraph is also
Quotations{\textless}/p{\textgreater}
\end{stylePreformattedText}

\begin{stylePreformattedText}
{\textless}/blockquote{\textgreater}
\end{stylePreformattedText}

\begin{stylePreformattedText}
{\textless}p{\textgreater}This paragraph is also Text body and has some
{\textless}em{\textgreater}text with emphasis
style{\textless}/em{\textgreater} and uses some
{\textless}b{\textgreater}hard
formatting{\textless}/b{\textgreater}.{\textless}/p{\textgreater}
\end{stylePreformattedText}

\begin{styleTextbody}
You can use your own Writer styles together with your own CSS style
sheet to create further style mappings, for example:
\end{styleTextbody}

\begin{stylePreformattedText}
{\textless}xhtml{}-style{}-map name={\textquotedbl}Some OOo
style{\textquotedbl} class={\textquotedbl}paragraph{\textquotedbl}
\end{stylePreformattedText}

\begin{stylePreformattedText}
 block{}-element={\textquotedbl}div{\textquotedbl}
block{}-css={\textquotedbl}block\_style{\textquotedbl}
\end{stylePreformattedText}

\begin{stylePreformattedText}
 element={\textquotedbl}p{\textquotedbl}
css={\textquotedbl}par\_style{\textquotedbl} /{\textgreater}
\end{stylePreformattedText}

\begin{styleTextbody}
to produce output like this:
\end{styleTextbody}

\begin{stylePreformattedText}
{\textless}div class=''block\_style''{\textgreater}
\end{stylePreformattedText}

\begin{stylePreformattedText}
 {\textless}p class=''par\_style''{\textgreater}Paragraph with Some OOo
style{\textless}/p{\textgreater}
\end{stylePreformattedText}

\begin{stylePreformattedText}
 {\textless}p class=''par\_style''{\textgreater}Yet
another{\textless}/p{\textgreater}
\end{stylePreformattedText}

\begin{stylePreformattedText}
{\textless}/div{\textgreater}
\end{stylePreformattedText}

\begin{styleTextbody}
Note that the rules for hard formatting are only used when
\textstyleUserEntry{xhtml\_ignore\_styles} is set to
\textstyleUserEntry{true}. It is not recommended to rely on these
rules, using real text styles is preferable. They are included because
the use of hard character formatting is very common even in otherwise
well{}-structured documents.
\end{styleTextbody}

\subsection{\label{ref:xhtmlfrontend}Using OpenOffice.org to create
XHTML documents}
\begin{styleTextbody}
The configuration file \textstyleUserEntry{cleanxhtml.xml} that is
distributed with Writer2LaTeX, can be used to create semantically rich
XHTML content, which can be formatted with your own stylesheet (you
should edit the file to add the URL to the stylesheet you want to use).
\end{styleTextbody}

\begin{styleTextbody}
A subset of the built{}-in styles in Writer are mapped to XHTML elements
(note that the style names are localized, so this is for the english
version of OpenOffice.org):
\end{styleTextbody}

\begin{longtable}[l]{|p{5.466cm}|p{5.466cm}|p{5.467cm}|}
\hline
\begin{minipage}[c]{5.466cm}\begin{styleTableHeading}
OOo Writer style
\end{styleTableHeading}
\end{minipage}&
\begin{minipage}[c]{5.466cm}\begin{styleTableHeading}
OOo Writer class
\end{styleTableHeading}
\end{minipage}&
\begin{minipage}[c]{5.467cm}\begin{styleTableHeading}
XHTML element
\end{styleTableHeading}
\end{minipage}\\\hline
\endhead
\begin{minipage}[c]{5.466cm}\begin{styleTableContents}
Text body
\end{styleTableContents}
\end{minipage}&
\begin{minipage}[c]{5.466cm}\begin{styleTableContents}
paragraph style
\end{styleTableContents}
\end{minipage}&
\begin{minipage}[c]{5.467cm}\begin{styleTableContents}
\textstyleSourceText{p}
\end{styleTableContents}
\end{minipage}\\\hline
\begin{minipage}[c]{5.466cm}\begin{styleTableContents}
Sender
\end{styleTableContents}
\end{minipage}&
\begin{minipage}[c]{5.466cm}\begin{styleTableContents}
paragraph style
\end{styleTableContents}
\end{minipage}&
\begin{minipage}[c]{5.467cm}\begin{styleTableContents}
\textstyleSourceText{address}
\end{styleTableContents}
\end{minipage}\\\hline
\begin{minipage}[c]{5.466cm}\begin{styleTableContents}
Quotations
\end{styleTableContents}
\end{minipage}&
\begin{minipage}[c]{5.466cm}\begin{styleTableContents}
paragraph style
\end{styleTableContents}
\end{minipage}&
\begin{minipage}[c]{5.467cm}\begin{styleTableContents}
\textstyleSourceText{blockquote}
\end{styleTableContents}
\end{minipage}\\\hline
\begin{minipage}[c]{5.466cm}\begin{styleTableContents}
Preformatted Text
\end{styleTableContents}
\end{minipage}&
\begin{minipage}[c]{5.466cm}\begin{styleTableContents}
paragraph style
\end{styleTableContents}
\end{minipage}&
\begin{minipage}[c]{5.467cm}\begin{styleTableContents}
\textstyleSourceText{pre}
\end{styleTableContents}
\end{minipage}\\\hline
\begin{minipage}[c]{5.466cm}\begin{styleTableContents}
List Heading
\end{styleTableContents}
\end{minipage}&
\begin{minipage}[c]{5.466cm}\begin{styleTableContents}
paragraph style
\end{styleTableContents}
\end{minipage}&
\begin{minipage}[c]{5.467cm}\begin{styleTableContents}
\textstyleSourceText{dt} (in \textstyleSourceText{dl})
\end{styleTableContents}
\end{minipage}\\\hline
\begin{minipage}[c]{5.466cm}\begin{styleTableContents}
List Contents
\end{styleTableContents}
\end{minipage}&
\begin{minipage}[c]{5.466cm}\begin{styleTableContents}
paragraph style
\end{styleTableContents}
\end{minipage}&
\begin{minipage}[c]{5.467cm}\begin{styleTableContents}
\textstyleSourceText{dd} (in \textstyleSourceText{dl})
\end{styleTableContents}
\end{minipage}\\\hline
\begin{minipage}[c]{5.466cm}\begin{styleTableContents}
Horizontal Rule
\end{styleTableContents}
\end{minipage}&
\begin{minipage}[c]{5.466cm}\begin{styleTableContents}
paragraph style
\end{styleTableContents}
\end{minipage}&
\begin{minipage}[c]{5.467cm}\begin{styleTableContents}
\textstyleSourceText{hr}
\end{styleTableContents}
\end{minipage}\\\hline
\begin{minipage}[c]{5.466cm}\begin{styleTableContents}
Citation
\end{styleTableContents}
\end{minipage}&
\begin{minipage}[c]{5.466cm}\begin{styleTableContents}
text style
\end{styleTableContents}
\end{minipage}&
\begin{minipage}[c]{5.467cm}\begin{styleTableContents}
\textstyleSourceText{cite}
\end{styleTableContents}
\end{minipage}\\\hline
\begin{minipage}[c]{5.466cm}\begin{styleTableContents}
Definition
\end{styleTableContents}
\end{minipage}&
\begin{minipage}[c]{5.466cm}\begin{styleTableContents}
text style
\end{styleTableContents}
\end{minipage}&
\begin{minipage}[c]{5.467cm}\begin{styleTableContents}
\textstyleSourceText{dfn}
\end{styleTableContents}
\end{minipage}\\\hline
\begin{minipage}[c]{5.466cm}\begin{styleTableContents}
Emphasis
\end{styleTableContents}
\end{minipage}&
\begin{minipage}[c]{5.466cm}\begin{styleTableContents}
text style
\end{styleTableContents}
\end{minipage}&
\begin{minipage}[c]{5.467cm}\begin{styleTableContents}
\textstyleSourceText{em}
\end{styleTableContents}
\end{minipage}\\\hline
\begin{minipage}[c]{5.466cm}\begin{styleTableContents}
Example
\end{styleTableContents}
\end{minipage}&
\begin{minipage}[c]{5.466cm}\begin{styleTableContents}
text style
\end{styleTableContents}
\end{minipage}&
\begin{minipage}[c]{5.467cm}\begin{styleTableContents}
\textstyleSourceText{samp}
\end{styleTableContents}
\end{minipage}\\\hline
\begin{minipage}[c]{5.466cm}\begin{styleTableContents}
Source Text
\end{styleTableContents}
\end{minipage}&
\begin{minipage}[c]{5.466cm}\begin{styleTableContents}
text style
\end{styleTableContents}
\end{minipage}&
\begin{minipage}[c]{5.467cm}\begin{styleTableContents}
\textstyleSourceText{code}
\end{styleTableContents}
\end{minipage}\\\hline
\begin{minipage}[c]{5.466cm}\begin{styleTableContents}
Strong Emphasis
\end{styleTableContents}
\end{minipage}&
\begin{minipage}[c]{5.466cm}\begin{styleTableContents}
text style
\end{styleTableContents}
\end{minipage}&
\begin{minipage}[c]{5.467cm}\begin{styleTableContents}
\textstyleSourceText{strong}
\end{styleTableContents}
\end{minipage}\\\hline
\begin{minipage}[c]{5.466cm}\begin{styleTableContents}
Teletype
\end{styleTableContents}
\end{minipage}&
\begin{minipage}[c]{5.466cm}\begin{styleTableContents}
text style
\end{styleTableContents}
\end{minipage}&
\begin{minipage}[c]{5.467cm}\begin{styleTableContents}
\textstyleSourceText{tt}
\end{styleTableContents}
\end{minipage}\\\hline
\begin{minipage}[c]{5.466cm}\begin{styleTableContents}
User entry
\end{styleTableContents}
\end{minipage}&
\begin{minipage}[c]{5.466cm}\begin{styleTableContents}
text style
\end{styleTableContents}
\end{minipage}&
\begin{minipage}[c]{5.467cm}\begin{styleTableContents}
\textstyleSourceText{kbd}
\end{styleTableContents}
\end{minipage}\\\hline
\begin{minipage}[c]{5.466cm}\begin{styleTableContents}
Variable
\end{styleTableContents}
\end{minipage}&
\begin{minipage}[c]{5.466cm}\begin{styleTableContents}
text style
\end{styleTableContents}
\end{minipage}&
\begin{minipage}[c]{5.467cm}\begin{styleTableContents}
\textstyleSourceText{var}
\end{styleTableContents}
\end{minipage}\\\hline
\begin{minipage}[c]{5.466cm}\begin{styleTableContents}
bold
\end{styleTableContents}
\end{minipage}&
\begin{minipage}[c]{5.466cm}\begin{styleTableContents}
hard formatting attribute
\end{styleTableContents}
\end{minipage}&
\begin{minipage}[c]{5.467cm}\begin{styleTableContents}
\textstyleSourceText{b}
\end{styleTableContents}
\end{minipage}\\\hline
\begin{minipage}[c]{5.466cm}\begin{styleTableContents}
italics
\end{styleTableContents}
\end{minipage}&
\begin{minipage}[c]{5.466cm}\begin{styleTableContents}
hard formatting attribute
\end{styleTableContents}
\end{minipage}&
\begin{minipage}[c]{5.467cm}\begin{styleTableContents}
\textstyleSourceText{i}
\end{styleTableContents}
\end{minipage}\\\hline
\begin{minipage}[c]{5.466cm}\begin{styleTableContents}
fixed pitch font
\end{styleTableContents}
\end{minipage}&
\begin{minipage}[c]{5.466cm}\begin{styleTableContents}
hard formatting attribute
\end{styleTableContents}
\end{minipage}&
\begin{minipage}[c]{5.467cm}\begin{styleTableContents}
\textstyleSourceText{tt}
\end{styleTableContents}
\end{minipage}\\\hline
\begin{minipage}[c]{5.466cm}\begin{styleTableContents}
superscript
\end{styleTableContents}
\end{minipage}&
\begin{minipage}[c]{5.466cm}\begin{styleTableContents}
hard formatting attribute
\end{styleTableContents}
\end{minipage}&
\begin{minipage}[c]{5.467cm}\begin{styleTableContents}
\textstyleSourceText{sup}
\end{styleTableContents}
\end{minipage}\\\hline
\begin{minipage}[c]{5.466cm}\begin{styleTableContents}
subscript
\end{styleTableContents}
\end{minipage}&
\begin{minipage}[c]{5.466cm}\begin{styleTableContents}
hard formatting attribute
\end{styleTableContents}
\end{minipage}&
\begin{minipage}[c]{5.467cm}\begin{styleTableContents}
\textstyleSourceText{sub}
\end{styleTableContents}
\end{minipage}\\\hline
\end{longtable}
\begin{styleTextbody}
So by using these styles only, you will create well{}-structured XHTML
documents. See the document \textstyleSourceText{sample{}-xhtml.sxw}
for an example of how to use this.
\end{styleTextbody}

\begin{styleTextbody}
Warning: Some elements are not allowed inside \textstyleSourceText{pre},
so this might in some cases lead to invalid documents. This will be
fixed in a later version of Writer2xhtml.
\end{styleTextbody}

\section{Using Writer2LaTeX from another Java application}
\begin{styleTextbody}
Writer2LaTeX uses the \textit{xmerge} framework. Please see the
documentation at
\href{http://xml.openoffice.org/xmerge}{http://xml.openoffice.org/xmerge}
for explanation about how to use Writer2LaTeX from Java.
\end{styleTextbody}
\end{document}
